\documentclass{notes}
\graphicspath{{../Images/}}

\fancyhead[l]{Jonathan Cheng}
\fancyhead[c]{MCB 55}
\fancyhead[r]{October 16, 2024}
\fancyfoot[c]{Page \thepage\ of \pageref{LastPage}}

\begin{document}

\section{Ebola Virus}

\begin{enumerate}
    \item Filoviridae Family
    \item RNA virus but has an unusual structure and genetic sequence
    \item Virions are variable in length (1000 nm average)
\end{enumerate}

5 species of Ebola are known so far

\image{Ebola Stats}{Ebola Stats}{Ebola statistics}{300pt}

Ebola virus is in BSL-4 because it can be extremely lethal

\image{Biosafety}{Biosafety}{Biosafety Levels}{300pt}

\subsection{Predicted Distribution}

3 species of Megachiroptera (bats) \textit{suspected} to reservoir Ebola virus in West / Central Africa

While direct evidence of Ebola in bats is limited, they are thought to play a significant role in the virus' ecology and transmission

Ebola has also been detected in the carcasses of chimpanzees, gorillas, and antelopes

\tab Infections in humans have been documented as occuring through the handling of bats, infected primates, and forest antelopes (dead or alive)

\section{Disease}

\subsection{First Ebola Outbreaks}

First outbreak of Ebola virus disease was reported in the DRC (in the village of Yambuku) in 1976

\tab 318 cases and 218 deaths (88\% fatality rate)

\tab Village had no running water, electricity, radio, phone, etc.

\subsubsection{Naming}

Yambuku village didn't want to be namesake

Congo already had disease named after it

Named after nearby river (Rio Ebola)

\subsection{Ebola Virus Disease (EVD)}

Symptoms typically appear 2-21 days after exposure to the virus, with an average onset of 8-10 days

Symptoms generally progress from `dry' to `wet' as the disease advances

\image{EVD}{EVD}{Ebola Virus Disease Symptoms}{230pt}

Previously named Ebola Hemorrhagic Fever

Primarily infects macrophages in the liver

\tab \indicates Infection in liver may be important due to fluid loss or hemorrhaging, leading to liver dysfunction

\tab Reduced blood clotting factors

\tab Low blood pressure (hypotension)

\subsubsection{Mortality}

Pathogenesis of the virus is closely tied to triggering vigorous inflammation

Does the high amount of virus replication cause the disease symptoms (systemic inflammation) directly

\textbf{OR}

Is there a potential role for immune initiated inflammation triggering hemorrhagic disease

\subsection{Course of Disease}

\begin{enumerate}
    \item Exposure
    \item 2-21 days incubation period (8-10 days average)
    \item Symptoms begin (virus become transmissible)
    \item Weakness, fever (1-3 days)
    \item Vomiting, diarrhea, hypotension (4-7 days)
    \item Confusion, bleeding, shock (7-10 days)
    \item Recovery or death
\end{enumerate}

Ebola outbreaks determined as concluded after 42 days of negative tests

Clearance of the virus may be delayed in a few immunologically protected body compartments and fluids (ear, testis, eye, brain, uterus, joint)

\subsection{Transmission}

Ebola virus is transmitted by direct contact with

\begin{enumerate}
    \item Blood
    \item Organs
    \item Bodily fluids (Saliva, urine, feces, vomit, sperm, sweat)
    \item Contaminated Objects (fomites)
    \item Infected animals
\end{enumerate}

\subsubsection{Human-to-Human Transmission}

\begin{enumerate}
    \item Close family contacts or caregivers
    \item Burial ceremonies with direct contact
    \item Hospital settings (Nosocomial transmissions)
\end{enumerate}

\subsubsection{Nosocomial Infections}

Healthcare workers are primarily infected through inadequate protection

Outbreak may spread to other patients in a hospital

Nosocomial infections of Ebola are higher than with most pathogens

\section{Outbreaks}

From 1976 to 2020, over 28 outbreaks of Ebola virus occurred in Africa

\tab Usually 1 to a few hundred people infected

\tab 25-90\% fatality, most often with Ebola Zaire

\subsection{2014 Epidemic}

The Guinea outbreak strain was Ebola-Zaire virus but a distinct strain from those in prior outbreaks

Seemed to be resolving after the first 8 weeks, but the outbreak of less than 100 increased exponentially, spreading to other countries

Why was this outbreak so much larger?

Through rural spread, there were 1,850 cases from 1974-2014 with 1200 deaths

Through urban spread the 2014 epidemic had 28,616 cases with 11,310 deaths

\tab Although there was only 40\% mortality

\subsubsection{Challenges}

\begin{enumerate}
    \item Never had Ebola outbreaks in so many countries at the same time
    \item It took 6 months for government and aid organizations to get involved
    \item Nonspecific treatment
    \item Very big structures had to be built, scaring patients and families
    \subitem Some governments turned to authoritarian tactics to force patients into compliance, scaring them further
    \item Most aid organizations were inexperienced with ebola
    \item Coordinating organizations in multiple places was extremely challenging
\end{enumerate}

\section{Immune Response}

Adaptive immune response is activated and hels resolve infection in those who survive

\tab Those who survive infection are thoguht to be protected from re-infection

IgG antibodies seem to be the key to protection

\tab Passive transfer of antibodies was also beneficial to people with active infection

\subsection{VSV Vaccine}

Vesticular Stomatitis Virus (VSV), which affects cattle, is weakened and acts as a vector

\tab Infected humans may be asymptomatic or have a mild fever for 2-5 days

The VSV displayed the surface proteins of Ebola virus, mounting an immune response without the dangers of infecting the Ebola Virus

\subsubsection{Vaccine Trial During 2014-2016 Outbreak}

Ring vaccination method

\textgreater \ 3500 individuals were recruited

\tab One group was vaccinated immediately, whereas the other group was vaccinated after 21 days

No cases were observed in the immediate cluster compared with 10 cases of EVD in the delayed group

Vaccine was improved in 2019

\subsubsection{Outbreak Monitoring}

Small outbreaks must be followed up with treatment and contact tracing

\tab Easier and cheaper than vaccinating the entire African population

March 2016

\begin{enumerate}
    \item 8 cases reported in Guinea
    \item 1000 contacts identified
    \item 800 vaccinated
\end{enumerate}

June 2016

\begin{enumerate}
    \item WHO declared end of Ebola virus in Guinea and Liberia
\end{enumerate}

\section{Treatment}

Symptoms of Ebola virus disease (EVD) are treated as they appear

\begin{enumerate}
    \item Fluids
    \item Electrolytes
    \item Oxygen
    \item Medication (Blood pressure, vomiting, diarrhea, fever, pain)
    \item Treating other infections
\end{enumerate}

\subsection{Antiviral Drugs}

Several drugs have been developed and approved for EVD, particularly for Zaire ebolavirus

\tab Monoclonal antibodies are now used successfully to treat patients with confirmed Ebola (90\% when administered soon after infection)

\tab Remdevisir was being developed to stop ebolavirus from making copies of itself but was shown to be effective against SARS-CoV-2

\subsection{Better Global Handling of Outbreaks}

\begin{enumerate}
    \item Ebola vaccine
    \item Rapid diagnostic tests
    \item Ebola treatment centers
    \item Monoclonal antibody therapies
    \item Public health education
    \item Contact tracing
    \item Quarantine measures
\end{enumerate}

\end{document}