\documentclass{notes}
\graphicspath{{../Images/}}

\fancyhead[l]{Jonathan Cheng}
\fancyhead[c]{MCB 55}
\fancyhead[r]{October 21, 2024}
\fancyfoot[c]{Page \thepage\ of \pageref{LastPage}}

\begin{document}

\section{Herpesviruses}

Herpesviruses existed 250 million years ago (the age of pangaea)

Herpesviruses can be found in every mammal species, as well as some other host species

\image{Herpesviruses}{Herpesviruses}{Herpesviruses are ubiquitous}{250pt}

\subsection{Evolution}

Herpesviruses are `specialist' viruses

\tab Most herpesviruses are extremely specialized to their specific host

Some herpesviruses have coevolved with their hosts

\tab \indicates Ancient (50m years) HSV1 (Human Herpesvirus Simplex 1) still  causes cold sores today

Some herpesviruses switch hosts

\tab \indicates HSV2 (5m years) is specialized to humans and none of the other species in the same families

\subsection{The Virion}

\begin{enumerate}
    \item Relatively large virus
    \item Genome is made of DNA
    \item Encode 70-200+ proteins
\end{enumerate}

\subsection{Generalities}

Herpesviruses are successful pathogens and extremely well adapted to their hosts

\tab Few or no clinical symptoms

\tab High infection rates within their host population (generally 60-80\%)

\tab Infection is life-long

\subsection{Modes of Life Cycle}

Latency: Dormant (very little expression / no virions produced)

Lytic cycle: Reactivation (virions are made / cell death)

After primary infection, herpesviruses persist for life in a latent stage

\subsubsection{Latent Stage}

\begin{enumerate}
    \item Upon infection, the viral DNA is transported to the nucleus where is circularizes
    \subitem Circular form is called an episome
    \item The episome is maintained in the nucleus of the infected cell with minimal viral expression
\end{enumerate}

\subsubsection{Lytic Replication (Reactivation)}

\begin{enumerate}
    \item Maintenance of the viral reservoir in the host
    \item Viral spread to new hosts
    \item Not associated with disease, but may be accompanied by clinical symptoms
    \item Stimuli leading to reactivation are not well understood
\end{enumerate}

\subsection{Evading Immune Response}

Can evade both innate and adaptive immune response

\tab Replication extremely quickly before cytotoxic T lymphocytes (CTLs) are activated

MHC-I molecules present viral peptides to CTLs

\tab Viral protein `hides' the MHC molecules in order to make the infected cell `invisible'

\tab A large fraction of their genome is dedicated to immune evasion

\section{Human Herpesviruses}

8 species from 3 subfamilies

\textbf{Alpha Herpesviruses} (Latency in neurons)

Herpes Simplex Virus 1 (HSV1) - 60\%

\tab Cold Sores

Herpes Simplex Virus 2 (HSV2) - 12\%

\tab Genital herpes

Varicella Zoster Virus (VSV) - 95\% (Vaccine)

\tab Chicken pox, Shingles

\textbf{Beta Herpesviruses} (Latency in CD34+ (Hematopoietic stem cells))

Cytomegalovirus (CMV) - \textgreater 50\%

\tab Retinitis, Birth defect

Human Herpesvirus 6 (HHV6) - 90\%

\tab Roseola

Human Herpesvirus 7 (HHV7) \ 85\%

\tab Roseola

\textbf{Gamma Herpesviruses} (Latency in \(\beta\) cells (Antibody-secreting cells))

Epstein-Barr Virus (EBV) - 90\%

\tab Mono, Lymphoma, Carcinoma

Kaposi's Sarcoma Associated Herpesvirus (KSHV) - 10\%

\tab Sarcoma

Herpesviruses do not survive long outside a host (transmission usually requires intimate contact)

\subsection{Human Cytomegalovirus (HCMV)}

Seroprevalence is high worldwide

Virus is spread through bodily fluids, including blood, urine, saliva, breast milk, tears, semen, vaginal fluids

CMV infection is generally asymptomatic

\tab Reactivation happens in episodes and is also generally asymptomatic

\tab \indicates Individuals are unlikely to know that they have been infected or are shedding

\subsubsection{Immunocompromised Patients}

Includes people:

\begin{enumerate}
    \item With acquired immune deficiency syndrome (AIDS)
    \item Who have receinved chemotherapy, radiation therapy, or steroid therapy
    \item Who have receinved an organ or bone marrow transplant
\end{enumerate}

Serious symptoms affect the eyes, lungs, liver, esophagus, stomach, and intestines

In transplant patients, may cause graft disfunction or rejection

\subsection{Congenital CMV}

\image{CMV Burden}{CMV Burden}{Relative burden of congenital CMV with relation to familiar congenital disorders}{300pt}

A pregnant woman can pass CMV to her fetus

\tab 1 in 150 babies are born with CMV

\tab \tab First children are less likely to have congenital CMV

\tab 10\% are symptomatic and will develop long-lasting problems

\tab 90\% are asymptomatic, some may still develop long-lasting problems

\subsubsection{Outcomes of Symptomatic Congenital CMV Infection}

\begin{enumerate}
    \item Visual impairment
    \item Brain abnormalities
    \item Microcephaly
    \item Hearing impairment
    \item Premature birth / Low birth weight
    \item Coordination disorders
    \item Liver, lung, and spleen issues
    \item Epilepsy
\end{enumerate}

\subsubsection{Ways to Help Prevent CMV}

\begin{enumerate}
    \item Do not share food, utensils, drinks, or straws
    \item Do not put a pacifier in your mouth
    \item Avoid contact with saliva
    \item Do not share a toothbrush
    \item Wash hands
\end{enumerate}

\subsection{Antiviral Drugs}

\image{Antiviral Drug}{Antiviral Drug}{Antiviral drugs inhibit the viral polymerase (nucleoside analogs)}{200pt}

Drugs are given to immunosuppressed patients with active HCMV replication

\tab May improve hearing and developmental outcomes for babies with signs of congenital CMV infection at birth

\tab Can have serious side effects

\subsection{Vaccine Development}

CMV vaccine ranked as highest priority by the US Institute of Medicine

Primary population that could benefit are women of child-bearing age and transplant recipients (prior to transplantation)

\tab Models suggest that vaccination of toddlers would offer strong indirect protection to women

\section{Herpes Simplex Virus 1 and 2}

HSV-1 causes sores around the mouth and lips (Can cause genetal herpes in 10-30\% of cases)

HSV-2 causes sores around genitals and rectum (Can cause oral herpes in 5\% of cases)

HSV also associated with recurrent eye infections and in rare cases encephalitis

\subsection{Transmission}

HSV is transmissible when areas of skin with the virus come into contact with mucous membranes (most common in mouth, vagina, and anus)

HSV-1 often spread through kissing or oral sex

HSV-2 often spread through vaginal, anal, or oral sex

\subsection{Seroprevalence}

3.7 billion people under age 50 has HSV-1

417 million people between ages 15 and 49 have HSV-2

\subsection{Latency}

Approximately 75\% of patients with primary genital HSV infection are asymptomatic

Establishment:

\begin{enumerate}
    \item Lytic replication in epithelial cells at the mucosal membrane
    \item Viral capsid moves down the axon via retrograde transport
    \item Latent HSV DNA in sensory neurons
\end{enumerate}

Reactivation works in the opposite way as establishment and causes an recurrent infection at same site

\tab Reactivation leads to shedding, transmission, and symptoms

Observed reactivation triggers:

\begin{enumerate}
    \item Mestruation
    \item Fatigue
    \item Stress
    \item Illness
    \item Exposure to sunlight
    \item Weakened immune system
\end{enumerate}

Latency is established in the Trigeminal ganglia (TG)(HSV-1), and the lumbar and sacral dorsal root ganglia (DRG)(Primarily HSV-2)

Asymptomatic shedding is frequent

\tab HSV-2 Shedding detected in \(\sim\)20\% of days in symptomatic individuals and \(\sim\)10\% of days in asymptomatic individuals

\tab Most transmission events occur when individuals are asymptomatic

\subsection{Drugs and Vaccine}

Vaccine candidates are currently being evaluated:

\begin{enumerate}
    \item For therapeutic purposes (reduction of viral shedding)
    \item For preventive purposes (prevent infection)
\end{enumerate}

Antiviral Drugs (Acyclovir, Valacyclovir, Famcyclovir) are nucleoside analogs

\tab Only \(\sim\)50\% reduction in transmission rate (\(\sim\)70-80\% suppression of lesions)

\subsection{Testing}

Testing for genital herpes is recomended for people who have symptoms

\begin{enumerate}
    \item Confirm infection
    \item Learn about medications
    \item Learn how to lower risk of spreading the infection
\end{enumerate}

CDC recommends against screening asymptomatic people

\begin{enumerate}
    \item Many false positives
    \item Oral vs. genital herpes cannot be determined
    \item No evidence that a blood test would change their sexual behavior
    \item Risk of stigmatizing people outweighs health outcomes
\end{enumerate}

\section{Gamma Herpesviruses}

\subsection{Kaposi's Sarcoma}

Before 1980s: Very rare disease found mainly in older men, patients who had organ transplants, or African men

More cases in the early 1980s in Africa and in gay men with AIDS

\tab \indicates At the peak of the AIDS epidemic, 20\% of men with AIDS developed KS

Discovery of HIV in 1983-1984 showed that AIDS is caused by a retroviral ablation of CD4+ T cells

\subsubsection{Viral Spread}

KS By 1990, scientists built a compelling case for KS being caused by a sexually transmitted infection in people with AIDS (but not by HIV)

\tab More common if a person had contracted HIV through sexual contact (rather than another way, like blood transfusion)

Other groups of people with suppressed immune systems (such as transplant recipients) are also at risk

Very few agents are known to increase the risk of cancer by 100 times, let alone 20,000

\tab Most well-known case is infection by hepatitis B virus causing liver cancer

\tab \indicates Strengthened the case for KS being caused by infection

\subsubsection{Discovery}

Compared the DNA from healthy tissue against DNA from KS lesions

Led to discovery of KSHV or HHV8

\subsection{EBV and KSHV}

Eptein Barr virus and KSHV infections are usually asymptomatic but can sometimes cause cancers in infected cells

Epstein Barr virus:

\begin{enumerate}
    \item Mononucleosis (kissing disease) (not a cancer)
    \item Burkitt lymphoma (B-cells)
    \item Nasopharynx Carcinoma (epithelial cells) (associated with smoked fish)
\end{enumerate}

Kaposi's Sarcoma herpesvirus:

\begin{enumerate}
    \item Primary effusion lymphoma (B-cells)
    \item Kaposi's sarcoma (Endothelial cells)
\end{enumerate}

\subsection{Latency}

Viruses need to tether their episomes to the DNA of the host, otherwise the episome would be lost during cell division

During latency, EBV encodes EBNA-1 and KSHV encodes LANA

\tab Used to attatch viral episome to chromosomes

\tab HSV-1 doesn't need this protein because neurons do not replicate

\image{LANA}{LANA}{LANA protein}{240pt}

Required for:

\begin{enumerate}
    \item Episome maintenance
    \item Episome replication
    \item `Encouraging' cellular division (possibly the cause of cancers?)
    \subitem Allows for viral spread without reactivation
\end{enumerate}

\end{document}