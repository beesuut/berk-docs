\documentclass{notes}
\graphicspath{{../Images/}}

\fancyhead[l]{Jonathan Cheng}
\fancyhead[c]{MCB 55}
\fancyhead[r]{October 30, 2024}
\fancyfoot[c]{Page \thepage\ of \pageref{LastPage}}

\begin{document}

\section{Papilloma Viruses}

A group of over 300 viruses (200 affect humans)

Vast majority of HPVs are asymptomatic

\tab Some cause genetal / skin warts or carry a risk of becoming cancerous

\section{Human Papillomaviruses (HPVs)}

HPVs are the most commonly sexually transmitted infection in the US

\(>\)80\% of men and women will be infected with at least one type in their lives

\tab Most people are asymptomatic and will clear the infection, usually within 2 years

HPV causes:

\begin{enumerate}
    \item Virtually all cases of cervical cancer
    \item 95\% of anal cancers
    \item 70\% of throat cancers
\end{enumerate}

\subsection{Types of HPV}

\subsubsection{Low-Risk Papilloma}

Generally asymptomatic, but can lead to genital warts (HPV6 and HPV11)

\subsubsection{High-Risk HPV}

Can lead to extensive cervical dyspasia and certain cancers

\tab HPV16 and HPV18 cause 70\% of HPV-related cancers

\subsection{Initial Infection}

HPV enters during sex through small abrasions

Virus pushes cell hyper-replication to support its replication (cause of warts)

\(\sim\)90\% of infected people heal within 2 years

\subsection{DNA Integration}

Occasionally, HPV DNA is integrated into tumor cell DNA

\tab Not a normal part of the virus' life cycle

\tab 0.8\% of cases develop cancer

\subsection{Prevention}

Safe sex practices are not 100\% effective

Pap smear and HPV tests can detect abnormal cell proliferation in the cervix and determine the type of HPV

\subsection{Virion}

Small viruses (50 nm) with DNA genome

L1 protein capsid (no lipid envelope)

\subsubsection{HPV Vaccines}

Made of L1 protein to protect against common HPVs

\tab \indicates Naturally assembled to form virus-like capsids (empty virions)

\subsection{Genes Linked to Cancers}

Proto-oncogenes: Proteins that normally contribute positively to cell proliferation

\tab Mutations might prevent proto-oncogenes preventing them from being turned off

\tab \indicates Called \textbf{oncogenes}

Tumor suppressor genes: Protein that prevent the unwanted proliferation of mutant cells

\tab Mutation might render these genes ineffective

\subsubsection{HPV E6/E7 Oncogenes}

\image{E6E7}{E6E7}{HPV E6/E7 Oncogenes}{250pt}

E6 and E7 are part of the viral genome and become cancerous when integrated into the human cell genome

\subsection{Cervical Cancer}

Integration results in dysregulation of E6 and E7 oncogenes

\tab Integration is a genetic accident (Dead end for virus as it is no longer able to transmit to new host)

\tab Gives cells a selective growth advantage, but does not necessarily cause cancer immediately

Takes 15-20 years for cervical cancer to develop in women with normal immune systems (5-10 years with weakened immune system)

Can also cause oral cancers (mouth, tongue, oropharynx), anal cancers, vulvar and vaginal cancers, penile cancers

\end{document}