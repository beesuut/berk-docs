\documentclass{notes}
\graphicspath{{../Images/}}

\fancyhead[l]{Jonathan Cheng}
\fancyhead[c]{MCB 55}
\fancyhead[r]{October 18, 2024}
\fancyfoot[c]{Page \thepage\ of \pageref{LastPage}}

\begin{document}

\section{Arboviruses}

Arboviruses (arthropod-borne viruses) exist in many families of viruses

There are \textgreater \ 500 recognized arboviruses worldwide

\tab 150 are known to cause human disease

\tab Mosquitos kill the most people per year of any animal

\subsection{Common Types of Mosquitoes}

There are over 3000 mosquito species

The most dangerous mosquitoes are certain species of Anopheles, Aedes, and Culex

Culex Mosquitoes \textemdash \ West Nile Virus, Western Equine Encephalitis, Eastern Equine encephalitis

Anopheles Mosquitoes \textemdash \ Malaria

Aedes Mosquitoes \textemdash \ Zika, Chikungunya, Yellow Fever, Dengue

\subsection{Mosquito Anatomy and Feeding}

Only female mosquitoes bite

\tab Most female mosquitoes must blood-feed on a vertebrate host to produce eggs

\begin{enumerate}
    \item Female mosquito mouthparts form a long, piercing proboscis
    \item Males have feathery antennae and a proboscis not suited to piercing skin
    \item Both genders usualy feed on nectar
    \item While taking blood, the female injects saliva in the host which serve as an anticoagulant
    \subitem also induces an inflammatory response
\end{enumerate}

\subsubsection{Viral Transmission}

The virus circulates and multiplies in the mosquito's hemolymph (blood) for several days

\tab The virus then penetrates and infects the mosquito's salivary glands

\tab After an incubation period of 1-2 weeks, the infected mosquito can transmit viruses to humans and animals while taking blood meals

\subsubsection{Role of Mosquito Saliva in Arbovirus Transmission}

Mice inoculated with West Nile Virus mixed with mosquito salivary gland extract had enhanced viremia compared to those inoculated without the salivary gland extract

\subsection{Arbovirus Transmission}

\begin{enumerate}
    \item In general, the arthropod host has no apparent disease
    \item Infection persists for life in the arthropod vector
    \item Virus can be transmitted transovarially to vector progeny
    \item Birds, rodents, and reptiles can be reservoir hosts
    \item Mamals can be `dead end hosts'
    \subitem Ex: Humans, horses
    \item Humans can be the primary vertebrate host for:
    \subitem Chikungunya, Dengue, Yellow Fever, Zika
\end{enumerate}

\section{West Nile Virus}

\subsection{History}

\begin{enumerate}
    \item Isolated from a feverish patient from the West Nile district of Northern Uganda in 1937
    \item Several `large' outbreaks occurred sporadically since the 50s, including epidemics in Russia, Spain, South Africa, etc.
    \item WNV is not entirely absent from Asia, but it is generally less significant than in other regions
\end{enumerate}

\subsection{Transmission Cycle}

Generally between birds through Culex mosquito vector

\tab Dead end hosts include humans and horses

West Nile arrived recently in the US and spread quickly

\image{WNV US}{WNV US}{West Nile Virus in the US}{260pt}

Almost all cases occur between June and October (mosquito season)

\subsection{Pathogenesis}

\begin{enumerate}
    \item Infection
    \item Migration to lymph nodes
    \item Migration to spleen (visceral-organ dissemination phase)
    \item Crossing the blood-brain barrier
    \item Infection of neurons
\end{enumerate}

75\% of infections are asymptomatic

25\% develop West Nile fever

Less than 1\% lead to neuroinvasive disease

\subsection{Vaccine}

Who is there a WNV vaccine for horses but not for humans?

\begin{enumerate}
    \item Unpredictable outbreaks (timing, geographic)
    \item Economic considerations (market size)
    \item Lack of progression to phase 3 trials (enrollment, case counts)
    \item Regulatory challenges
\end{enumerate}

\subsection{Risk Reduction}

Vector management (Educate, Surveillance, Mosquito Control)

Educate: Eliminate standing water

Surveil: Finding and monitoring places where adult mosquitoes lay eggs

\tab Tracking the viruses mosquito populations may be carrying

\tab Allow chickens to be bitten by mosquitoes and check immune response

Mosquito Control: Mosquitofish live in ponds and eat mosquito larvae

\tab Use of Mosquito Dunks which is added to standing water and kills larvae

\tab \indicates Biological larvacide called BTI (Bacillus thuringiensis subspecies israelensis)

\tab Fumigation if BTI fails

\end{document}