\documentclass{notes}
\graphicspath{{../Images/}}

\fancyhead[l]{Jonathan Cheng}
\fancyhead[c]{MCB 55}
\fancyhead[r]{October 4, 2024}
\fancyfoot[c]{Page \thepage\ of \pageref{LastPage}}

\begin{document}

\section{Zoonosis}

Word Origin - Greek

\begin{enumerate}
    \item Zoon - Animals
    \item Noson - Disease
\end{enumerate}

\section{What Are Zoonoses?}

Infectious diseases that are transmissible from vertebrate animals to humans under natural conditions (WHO)

Zoonotic Spillover: The event in which a pathogen jumps from animal to human (or vice versa)

Zoonoses represent a major public health problem worldwide

\begin{enumerate}
    \item 60\% of known infectious diseases in people are spread from animals
    \item Significant morbidity and mortality
    \item Impacts regional / global economies
\end{enumerate}

\subsection{Is Dengue a Zoonosis?}

Infected person \indicates infected mosquito (\textit{Aedes aegypti}) \indicates healthy person

\tab \indicates Began as a zoonosis (from monkeys) but is now endemic to humans


\section{Characteristics of Zonotic Pathogens}

\begin{enumerate}
    \item Can be stably established in animal populations
    \item Can transmit from animals to people with little or no subsequent person-to-person transmission
    \subitem Ex: West Nile virus / Rabies virus
    \item Can spread efficiently between people once introduced from an animal reservoir
    \subitem Leading to epidemic (Ebola virus)
    \subitem Or pandemic (Pandemic influenza / Coronaviruses)
\end{enumerate}

\section{Transmission routes}

\begin{enumerate}
    \item Fecal Oral (Hantavirus)
    \item Inhalation (Coronaviruses)
    \item Contact with body fluids (Nipah virus)
    \item Penetrating wounds (Rabies virus)
    \item Vector transmitted — Mosquitoes and ticks (West Nile virus)
    \subitem Zoonotic because birds are host reservoir
\end{enumerate}

\subsection{Reverse Zoonoses}

As of September 2022, the human SARS-CoV-2 virus has been detected in 25 animal species

\section{Nipah Virus}

Endemic to fruit bats in Southeast Asia

\tab \indicates Virus doesn't affect bats, but they are carriers and can spread it through bodily fluids (saliva or urine)

\subsection{Malaysia - 1998-1999}

\begin{enumerate}
    \item Transmission facilitated by intermediate hosts
    \item Infection of pigs
    \subitem Pig-to-pig \indicates pig-to-human by aerosol
\end{enumerate}

\section{Disease}

Most cases are symptomatic

Fatality of 40-100\%

Causes severe, rapidly progressive encephalitis

\tab \indicates May have delayed onset or relapse, months or years after infection

Can cause respiratory involvement or respiratory illness

\tab \indicates May cause acute respiratory distress syndrome

Can spread through:

\begin{enumerate}
    \item Consumption of contaminated food or fruit products such as raw date palm juice
    \item Close contact with an infected person's bodily fluids
\end{enumerate}

\subsection{Bangladesh}

Nearly annual outbreaks in Bangladesh

Solved with rudimentary nets above date palm trees

\image{Date Palm}{Date Palm}{Date palm nets}{180pt}

\section{Factors That Influence Spillover Events}

\subsection{Reservoir Host Distribution}

Affected by interactions between species:

\begin{enumerate}
    \item Geographic range overlap
    \item Travel
    \item Hunting, trade, and consumption of meat from wild species
\end{enumerate}

\subsection{Reservoir Host Density}

Ex: Hantaviruses transmitted through deer mice (10-12\% carriers) in Yosemite deluxe cabins

\tab \indicates Deer mice burrowing in styrofoam insulation

Cause diseases such as:

\begin{enumerate}
    \item Hantavirus pulmonary syndrome (HPS)
    \item Hemorrhagic fever with renal syndrome (HFRS)
\end{enumerate}

\subsection{Animal Host-Associated Factors}

Bats are often reservoirs because their immune system can deal with a wide variety of pathogens

\tab \indicates Tuned down immune response because of flight adaptations

Ecological habits of animals which frequently share the environment with humans and domestic / livestock animals

Some animals can act as intermediate hosts in spillover events

\subsection{Human Host-Associated Factors}

\begin{enumerate}
    \item Immunological Factors
    \item Genetic Factors
    \item Behavioral Factors
\end{enumerate}

\subsection{Phylogenetic Distance Between Host Species}

Risk of spillover is higher among species with greater phylogenetic proximity

Phylogenetically distant hosts may cause more virulence

\subsection{Characteristics of the Pathogens}

`Generalist' pathogens (as opposed to `specialist') have the ability to infect a broad host range and are more able to jump the barrier between species

Different viral taxonomic groups have varied zoonotic potential

\image{Arms Race}{Arms Race}{Viruses and hosts evolve to survive}{270pt}

\subsection{Environmental Factors}

\begin{enumerate}
    \item Loss of biodiversity is associated with emergence and spread of infectious disease
    \item Change in land use
    \item Global warming modifies the behavior of reservoir species and recipient hosts
\end{enumerate}

\section{The One Health Challenge}

The health of people is closely connected to the health of animals and our shared environment

\tab \indicates Heath issues need to be fought at the human-animal-environment interface

\section{Bats}

Many different viral families are found in bats, but most don't appear to make them sick

\image{Bat Viruses}{Bat Viruses}{Viruses found in bats}{300pt}

\subsection{SARS-CoV-2}

Seventh coronavirus known to infect humans

\tab \indicates Generally begins in bats and goes through an intermediate host

\image{Bat Spillover}{Bat Spillover}{Coronaviruses in humans}{150pt}

\subsection{Why are Bats Special?}

\begin{enumerate}
    \item Second most diverse mammalian order on Earth after rodents and are gregarious (living in social colonies)
    \subitem Facilitates rapid transmission of pathogens
    \subitem Large populations could sustain acute-immunizing infections
    \subitem Selects for viruses that can adapt to novel host environments
    \item Frequent interaction with humans
    \subitem Peridomestic habits
    \subitem Bushmeat
    \subitem Deforestation
    \item Relatively long lifespan (3-10x longer than equivalently-sized mammals)
    \subitem Facilitate viral persistence for chronic infections
    \item Fly long distances
    \subitem Allows dispersal over long distances
    \subitem Flight may mimic fever
    \item The classical pathology caused by strong activation of the immune system in response to viral infection that is seen in humans (cytokine storm) does not occur in bats
    \subitem Weakened DNA sensing
    \subsubitem Due to the high metabolic requirement of flight
    \subitem Some aspects of innate immune system are always `on'
\end{enumerate}

\section{Animal-Human Transmission}

Most animal viruses are unable to replicate in the human body

\tab \indicates Of those that can, fewer are transmissible between humans or human-exclusive

\subsection{Obstacles to Replication in New Hosts}

\begin{enumerate}
    \item Needs to interact with many different cellular proteins in order to enter a new cell and replicate
    \item Needs to evade immune responses of the host
\end{enumerate}

\subsubsection{Viral Life Cycle}

\begin{enumerate}
    \item Attachment
    \item Penetration
    \item Uncoating
    \item Replication
    \item Assembly
    \item Release
\end{enumerate}

\image{Viral Life Cycle}{Viral Life Cycle}{Viral life cycle}{360pt}

Viruses exploit host cell machinery for all aspects of their multiplication

\tab \indicates Animal viruses will only replicate in humans if they can interact with all the useful proteins that they need

\subsection{Viral Sensing}

Cells monitor their intra and extracellular spaces for the presence of atypical nucleic acid (wrong location or unusual structure) associated with viral infection

\begin{enumerate}
    \item Cytosolic DNA
    \item Double-stranded RNA
    \item Unusually capped mRNA
\end{enumerate}

\subsubsection{Interferon Pathway}

When a cell is infected, viral PAMPS are recognized by specific receptors

\tab \indicates Initiates events, untimately resulting in the infected cells releasing a small set of molecules called interferons (IFN)

\tab \indicates Cells which detect IFN will start making proteins whose function is to combat viral infection

To replicate within a host, viruses need to inhibit the IFN response

\tab \indicates Hiding dsRNA (replicating genome) is one strategy used to prevent being detected and eliminated

\image{IFN Pathway}{IFN Pathway}{IFN Pathway}{430pt}

Ex: OAS / RNase L pathway was one of the first ISG (interferon stimulated gene) antiviral mechanisms to be identified

\tab \indicates RNase indiscriminantly destroys RNA

\subsubsection{Infecting New Hosts}

Often, pathogens must adapt to successfully infect a novel host

\begin{enumerate}
    \item Use different cell surface receptors
    \item Escape a novel type of immune response
    \item Ensure they are transmitted by the new host
\end{enumerate}

RNA viruses are the most likely to be associated with spillover events because of their

\begin{enumerate}
    \item High mutation rates
    \item High multiplication rates
\end{enumerate}














\end{document}