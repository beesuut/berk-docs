\documentclass{notes}
\graphicspath{{../Images/}}

\fancyhead[l]{Jonathan Cheng}
\fancyhead[c]{MCB 55}
\fancyhead[r]{October 2, 2024}
\fancyfoot[c]{Page \thepage\ of \pageref{LastPage}}

\begin{document}

\section{Viruses}

Viruses outnumber cellular life

Estimated \(10^{31}\) viruses (greatest biodiversity on earth)

\(>95\%\) of ocean biomass is microbial and about 20\% of ocean microbes are destroyed each day by viruses

\tab \indicates Major player in carbon / oxygen cycles regulating the atmosphere

Bacteriophages that infect \textit{vibrio cholera} may be an important factor in regulating cholera outbreaks and pandemics

Viruses can infect all types of life forms (animals, plants, bacteria, archaea)

\tab \indicates Every living organism studied has at least 1 virus associated with it

\subsection{Types of Viruses}

\begin{enumerate}
    \item Zoonosis
    \subitem Coronavirus
    \subitem Rabies
    \subitem Ebola
    \item Forever Viruses (Infect more than 50\% of the population)
    \subitem HSV1 (Herpes Simplex Virus 1)
    \subitem HIV (Human Immunodeficiency Virus)
    \item Viruses and Cancers
    \subitem HBV (Hepatitis B)
    \subitem HCV (Hepatitis C)
    \subitem HPV (Human Papilloma Virus)
\end{enumerate}

\subsection{Drug Resistance}

Viruses are becoming multi-drug resistant

\tab \indicates Bacteriophages may be used when antibiotics become ineffective

Ex: Clinical use of bacteriophage therapy for \textit{Mycobacterium chelonae} infection

\subsection{Coevolution}

8\% of our genome is virus (LTR retro-transposons)

\tab \indicates Retroviruses inject their own genome into hosts

\subsection{Syncytin-1}

Endogenous retroviral envelope protein

\begin{enumerate}
    \item Derived from ancient virus
    \item Retained its fusogenic properties
    \item Participates in trophoblast fusion and the formation of a syncytium during placenta morphogenesis
\end{enumerate}

\image{syncitin-1}{syncitin-1}{Use of syncitin-1 during pregnancy}{300pt}

\subsection{Tobacco Mosaic Disease: A Mysterious Infection (1886)}

The agent can be transferred between plants (A bacteria? A toxin?)

1892: Found capable of permeating porcelain Chamberland filters (something bacteria could not do)

\tab \indicates Diatomaceous earth filters which hold back all bacteria

1898: Martinus Beijerinck reevaluated Dmitri Ivanovsky's findings

\subsubsection{The Tobacco Mosaic Virus (TMV)}

1935: TMV was the first virus to be visualized using electron microphotograph

\tab By Wendell Stanley (Namesake of Stanley Hall)

\subsection{What is a Virus?}

\begin{enumerate}
    \item Infectious, obligate parasite
    \item Comprises genetic material (DNA or RNA)
    \item Surrounded by a protein coat and/or envelope derived from host cell membrane
\end{enumerate}

\subsubsection{Viruses have vast differences in size}

Forms:
\begin{enumerate}
    \item Virion: Form of virus outside cell
    \item Infected Cell
\end{enumerate}

\image{Virion}{Virion}{Virion vs. Infected Cell (Not to scale)}{\textwidth}

\newpage

\subsection{Stages of Viral Infection}

\begin{enumerate}
    \item Attatchment
    \item Penetration
    \item Uncoating
    \item Replication
    \item Assembly
    \item Release
\end{enumerate}

\image{Stages Of Infection}{Stages}{Stages of a viral infection}{220pt}

Hundreds of progeny virions can be produced from a single infections virus particle

\subsection{Viral Genome}

Viruses have about 1 million times less genetic information than the genomes of most plants and animals they effect

Viruses must do a lot with only a small amount of genetic information

\begin{enumerate}
    \item Take over cellular machinery to mass-produce viral proteins (replicate viral genome)
    \item Avoid eradication by immune system
    \item Keep infected cell alive long enough to complete replication cycle
\end{enumerate}

\subsubsection{Mutations}

Viruses mutate \textbf{a lot}

\image{Mutation Rate}{Mutation}{Quasi-species adaptability}{230pt}

The phenotype of the mutation can be neutral, advantageous, or deleterious

More mutations per genome \indicates More chances for a deleterious protein

\image{Mutations Per Genome}{Per Genome}{Model of Error Catastrophe}{230pt}

Molnupiravir: Antiviral forcing SARS-CoV-2 to mutate more (`lethal mutagenesis')

\tab \indicates Error Catastrophe

\subsection{Not All Viruses are Bad?}

\begin{enumerate}
    \item Some grasses can only grow at \(>\) 50\degree C when infected with a virus
    \item Herpesviruses can provide protection against \textit{listeria} infection
    \subitem Immune system is already activated
\end{enumerate}

\section{Wider Connections}

Viruses are important to the study of molecular and cell biology by providing simple systems which can manipulate and investigate the functions of cells

\begin{enumerate}
    \item Tumor suppressors
    \item Proto-oncogenes
    \item DNA replication
    \item Gene expression
    \item Vaccines
    \item Splicing
    \item Gene therapy
    \item CRISPR
\end{enumerate}

\end{document}