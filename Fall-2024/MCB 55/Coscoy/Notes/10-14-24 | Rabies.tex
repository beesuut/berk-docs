\documentclass{notes}
\graphicspath{{../Images/}}

\fancyhead[l]{Jonathan Cheng}
\fancyhead[c]{MCB 55}
\fancyhead[r]{October 14, 2024}
\fancyfoot[c]{Page \thepage\ of \pageref{LastPage}}

\begin{document}

\section{Rabies}

\subsection{Timeline}

\begin{enumerate}
    \item First written record of rabies causing death in dogs and humans: 2300 B.C.
    \item Pliny the Elder makes a list of rabies cures: 79 A.D.
    \item Richard Mead publishes the opinion that rabies outbreaks are controlled by the moon: 1702
    \item First documented case in the US: 1753
    \item Rabies recognized as an infectious disease: 1804
    \item First rabies vaccines by Pasteur: 1885
\end{enumerate}

\subsection{Other Facts}

Lyssa: Derived from the greek word for `rage' and `fury'

Rabies kills about 60,000 people yearly

Transmitted by rabid animals such as:

\begin{enumerate}
    \item Dogs
    \item Bats
    \item Raccoons
    \item Foxes
\end{enumerate}

\subsection{Rabies Virus}

Rabies is caused by the rabies virus

\tab \indicates Type species for the Lyssavirus genus in the Rhabdoviridae family

\tab \indicates Enveloped RNA virus with a helical capsid

Other Lyssaviruses are also usually transmitted via a bite from an infected (rabid) animal

\tab \indicates Cause a similar illness known as rabies

All mammals can get rabies

\tab Non-mammal carnivores are less often associated with rabies cases

\tab \tab Small rodents are rarely transmitters because they die by bites from larger mammals

\tab \tab Non-mammals never transmit rabies

\subsection{Transmission}

Virus cannot enter intact skin

\tab \indicates Transmission is almost always a bite

\tab Occasionally saliva may come into contact with mucous membranes or fresh skin lesions

\tab Rarely, there may be aerosolized rabies or human-to-human through transplantation

40-60\% of animal bite cases are reported in children

\begin{enumerate}
    \item More susceptible to bites on face / scalp because of height
    \item More susceptible to play outside
    \item Cannot ward off animals easily
    \item More likely to provoke an animal
\end{enumerate}

\subsubsection{Control of Animal Vectors}

Primary strategy for the prevention of rabies in humans

Typically targets dogs through vaccination, management of stray populations, and castration

Rabies thought to be controlled in 70\% of dogs are vaccinated using inactivated virus vaccine

\tab \indicates Immunity lasts around 3 years


\subsection{Rabies Deaths by Country}

Extremely high in Africa and (Southeast) Asia

\(>55,000\) deaths worldwide

~98\% of cases caused by dog bites

\subsubsection{Delhi Vulture Crisis}

Widespread use of drugs such as diclofenac (nonsteroidal anti-inflammatory drug (NSAID)) in livestock

\tab \indicates Resulted in a substantial increase in the population of feral dogs

\tab \indicates Led to 38 million dog bites, 47,000 deaths, and \$34 billion in costs

\subsection{Rabies in the US}

In 1938, most cases of rabid animals were domestic (9,321) with some wild (44)

\tab \indicates The numbers have since reversed

Human rabies deaths are relatively rare

\begin{enumerate}
    \item 1-3 cases reported anually
    \subitem 25 cases (2009-2018) with several acquired outside of the US
    \item Over 90\% of animal rabies cases occur in wildlife (6,690 cases in 2009)
    \item Annual prevention costs ~\$300 million
\end{enumerate}

Most rabies exposure in the US are from infected bats (0.1\% of bats have rabies)

\tab Bite wound only the size of a hypodermic needle

\tab \indicates Therefore don't seek medical attention or PEP

\subsection{Rabies Control in Wildlife}

Trap / Vaccinate / Release (TVR)

\tab \indicates Effective in Canada raccoons (often combined with oral baits)

Oral baits with antivirus

\tab \indicates Has been effective in Europe and Canada

\tab \indicates Slowed outbreak in Ohio raccoons

\tab \indicates used in Texas for coyotes and foxes

\subsection{Pathogenesis}

\subsubsection{Course of Disease}

Incubation period for rabies is typically 2-3 months but may vary from 1 week to several years

\tab Depends on factors such as location of virus entry and viral load

Initial symptoms are not very specific

\begin{enumerate}
    \item Fever with pain
    \item Tingling
    \item Pricking
    \item Burning sensation (paraethesia)
\end{enumerate}

\image{Rabies Pathogenesis}{Pathogenesis}{Rabies Pathogenesis}{300pt}

2 forms in humans:

\begin{enumerate}
    \item Furious Rabies: Hyperactivity and hydrophobia
    \subitem Death after a few days due to cardio-respiratory arrest
    \item Paralytic rabies accounts for around 20\% of human cases
    \subitem Coma slowly develops and eventually dies
    \subitem Often misdiagnosed, due to under-reporting
\end{enumerate}

\subsubsection{Furious Rabies}

Rabies increases saliva production because virus is spread through saliva

Painful spasms develop in the muscles that control breathing and swallowing

\tab Feels like drowning \indicates Hydrophobia

Other Symptoms:

\begin{enumerate}
    \item Delirium
    \item Aggression
    \item Drooling
    \item Muscle Spasms
    \item Dizziness
    \item Hallucinations
\end{enumerate}

\subsection{Preventive Immunizations}

Any person who could be exposed to the live rabies virus

\begin{enumerate}
    \item Lab staff
    \item Veterinarians
    \item Animal / bat handlers
    \item Wildlife officials
\end{enumerate}

Children travelling to / living in a rabies-endemic area

Travellers who may be more than 24 hours from a medical center with a post-exposure vaccine

\subsection{Post-Exposure Prophylaxis (PEP)}

Immediate treatment of a bite victim after rabies exposure

Prevents virus entry into the central nervous system

Consists of:

\begin{enumerate}
    \item Extensive washing and local treatment of wound
    \item Course of potent and effective rabies vaccine
    \item Administration of rabies immunoglobulin (RIG)
\end{enumerate}

Why are there so many deaths by RABV when treatment is available and effective?

\begin{enumerate}
    \item Majority of victims can / do not receive rabies vaccination or do not complete the full course
    \item Use of rabies immunoglobulins (RIG) is very low
    \item Lack of awareness about the potential severity of animal bites
    \item Some victims cannot afford the cost of PEP or may resort to indigenous treatment practices
\end{enumerate}




\end{document}