\documentclass{notes}
\graphicspath{{../Images/}}

\fancyhead[l]{Jonathan Cheng}
\fancyhead[c]{MCB 55}
\fancyhead[r]{October 25, 2024}
\fancyfoot[c]{Page \thepage\ of \pageref{LastPage}}

\begin{document}

\section{Hepatitis C Virus}

Hepatitis means inflammation of the liver

\tab Most common types of viral hepatitis are A (vaccine), B (cancerous), and C (cancerous)

Related to other human pathogens (Yellow fever, Zika, Dengue)

\tab Spread globally in 20th century due to novel medical practices

HCV is the cause of hepatitis cirrhosis (liver inflammation) and liver cancer in infected individuals

\tab 71 million people (1\%) are currently living with Hepatitis C virus

\subsection{Transmission}

Blood contact

\begin{enumerate}
    \item Intravenous drug use
    \item Blood transfusions
    \item Contaminated medical / dental instruments
    \item Mother to baby (5\% risk)
    \subitem Higher risk if mother is HIV+
    \item Low risk of sexual transmission (if blood is involved)
    \item Unregulated tattoos or body piercings
    \item Sharing personal item (glucose montors, razors, \(\dots\))
\end{enumerate}

\subsubsection{Egypt}

Egypt is the country with the highest prevalence of the disease in the world (10\% of Egyptians had chronic hepatitis C in 2008)

Campaign to rid the country of schistosomiasis, but needles were reused

\subsection{Inflammation}

Can cause both acute and chronic infection

New HCV infections are usually asymptomatic, but acute hepatitis C occurs within the first 6 months after exposure

\tab \indicates Around 30\% of infected people clear the virus within 6 months of infection without treatment

\tab \indicates 70\% of HCV-infected individuals will develop chronic HCV infection

\tab \tab Of those, the risk of cirrhosis ranges between 15-30\% within 30 years

\subsection{Liver Transplantation}

Liver failure due to hepatitis C is one of the most common reasons for liver transplantation in the US

Chronic inflammation likely causes damage to cell DNA and affects growth / division, leading the the growth of tumors and cancer

\tab Not fully understood

\subsection{Estimates}

\begin{enumerate}
    \item 1.5 million new infections per year
    \item 80\% of infected people don't know
    \item Less than 10 million people are being treated
    \item Currently no effective vaccine against hepatitis C
    \item Direct-acting antiviral medicines (DAAs) can cure more than 95\% of cases, but access to diagnosis and treatment is low
\end{enumerate}

\section{Drugs and Vaccines}

\subsection{Viral Quasispecies}

Hepatitis C exhibits significant genetic variability and exists as quasispecies within infected individuals

\tab Quasispecies: a population of closely related but genetically diverse viral variants

Due to:

\begin{enumerate}
    \item High mutation rate of the error-prone RNA-dependent RNA polymerase (\(10^{-4}\) substitutions per site per replication)
    \item Rapid viral replication (\(10^{12}\) new virions produced daily)
    \item Selective pressures from host immune system
\end{enumerate}

Statistically, every genome that enters a cell, exits with 1 or more mutations

\subsection{Direct-Acting Antivirals}

Ribavirin and IFN-1

Work by targeting the virus directly, making them more effective than older treatments

\image{DAA}{DAA}{Direct-acting antivirals}{270pt}

Short-course oral treatment have few, if any side effects

\tab DAAs can cure most people within 8-12 weeks

\subsection{Elimination}

The WHO have committed to eliminate viral hepatitis by 2030

\tab Only 11 countries are on track, with another 24 expected by 2050

\tab Most countries, including the US, are unlikely to eliminate HCV by 2050

\subsection{HCV in the US}

The number of people infected with HCV in the US has increased dramatically

HCV disproportionately impacts minority and marginalized population

\begin{enumerate}
    \item Substance use disorders
    \item People of color
    \item Incarcerated persons
    \item People living in poverty
\end{enumerate}

Hepatitis C treatment requiresan antibody test to detect prior infection, followed by an RNA test to determine whether infection is active

Among those diagnosed, HCV coverage is far below what is needed to achieve elimination goals

\begin{enumerate}
    \item 1/3 of HCV-infected individuals with insurance are treated (lower for patients without insurance)
    \item High cost of DAAs
    \subitem \$90,000 per patient initially; still around \$20,000 (As low as \$60 in low- and middle-income countries)
    \subitem Requirement of sobriety
    \subitem Requirement to document evidence of liver fibrosis
    \subitem Access to treatment particular to field specialists
\end{enumerate}

Currently plans to make treatment free for patients on Medicaid, uninsured, in the prison system, or on a Native American reservation

\end{document}