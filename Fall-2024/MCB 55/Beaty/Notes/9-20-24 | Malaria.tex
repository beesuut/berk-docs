\documentclass{notes}
\graphicspath{{../Images/}}

\fancyhead[l]{Jonathan Cheng}
\fancyhead[c]{MCB 55}
\fancyhead[r]{September 20, 2024}
\fancyfoot[c]{Page \thepage\ of \pageref{LastPage}}

\begin{document}

\section*{Malaria}
Ancient disease and still one of the leading causes of mortality

Nobel Prizes:
\begin{enumerate}
    \item Human to human by mosquito
    \item Protozoan organism (Pasmodium)
    \item using DDT to eradicate malaria
    \item Artemisinin drug extracted from wormwood
\end{enumerate}

\subsection*{Malaria in the World}
Primarily (94\%) in Subsaharan Africa (Aprrox. 3.2 billion people at risk)

Around 249 million cases per year (84 countries with 608,000 malaria deaths)

80\% of deaths in children under the age of 5

\subsection*{Epidemiology}
One of the leading causes of death from infection disease

Why?
\begin{enumerate}
    \item Socioeconomic conditions
    \item Malaria accounts for up to 40\% of public health expenditures in some countries
    \item The vector: ``The long lifespan and strong human-biting habit of the African mosquito vector species is the main reason why more than 85\% of the world's malaria deaths are in Africa''
\end{enumerate}

\subsubsection*{Mosquito Vector}
Anopheles is the only mosquito genus capable of spreading malaria (only transmitted by females)

Transmission is seasonal in many places (Peaking during or just after rainy season because of presence standing water)

20 species of Anopheles can submit Plasmodium parasites but certain species transmit better

\tab \indicates Anopheles gambiae is a night biter that preferentially bites humans

\tab \indicates May bite up to 20 people in 1 night

\subsection*{Malaria in the US}
Major cause of morbidity in the 1800s into the Civil War

\tab 600,000 cases in 1914, primarily in Louisiana and Florida

Some species of Anopheles (A. albimanus, A. freeborni) that are present in the southern US

CDC says there are 1,700 imported cases of malaria each year in the US (Mostly from travelers acquiring infections in other regions)

\subsection*{Mosquito Vector Control - DDT}
A. gambiae was found in Brazil in 1930

Soper was brought in to eliminate A. gambiae after an epidemic in 1938

\tab \indicates Previously used diesel oil and Paris green to spray in water

DDT discovered in 1939 during WWII (Fred Soper saw potential as an insecticide to kill mosquitoes for up to a month)

\subsection*{Eradication (1947-1955)}
Combined systematic spraying of DDT and treatment of infected patients with chloroquine

\tab \indicates Very effective in some countries (Sri Lana went from 1 million cases in 1955 to 18 in 1963)

Campaign cost was \(\sim\)\$430 million, however eradication in sub-Saharan Africa was never fully implemented

\subsubsection*{End of First Global Malaria Campaign}
Cost was high and in 1963 the US Congress withdrew funding

Drug Resistance: Chloroquine resistance of Plasmodium species began to occur and spread

Problems with DDT:
\begin{enumerate}
    \item Widespread use of DDT in agriculture led to widespread DDT resistance
    \item DDT found to be a devastating environmental pollutant causing severe damage to wildlife
    \item Complicaation occured from DDT damaging an array of non-harmful insects
\end{enumerate}

\subsection*{Control Strategy in 1990s}

\begin{enumerate}
    \item ``Control'' rather than eradication
    \item Prompt treatment for all episodes of disease (within 24 hours of the onset of symptoms if possible)
    \item Bednet use combined with insecticides
    \item Indoor residual spraying to kill mosquitos that rest on walls and roofs
\end{enumerate}

\subsection*{Malaria US outbreak in June/July 2023}
Anopheles mosquitos have always been present but few infected individuals and mosquito control have stopped any big outbreaks since 1951

Only 8 cases in Florida and Texas (Few people, but shows the ability of malaria to return)

\subsection*{Intracellular Protozoan Parasite}
\begin{enumerate}
    \item Malaria is caused in humans by 4 species
    \subitem Plasmodium vivax
    \subitem P. falciparum
    \subitem P. ovale
    \subitem P. malariae
    \item Infects red blood cells (RBCs) which makes parasites relatively easy to find in blood
    \item By living in RBCs, the parasite evades key adaptive immune mechanisms (CTLs)
\end{enumerate}
\textbf{Parasites:} eukaryotic organisms living in other organisms
\begin{enumerate}
    \item Protozoa: Single cells
    \item Worms: Multicellular
\end{enumerate}

\subsubsection*{Naming}
Genus and Species for parasites and bacteria

Viruses only have species

\subsection*{Infection}
Sporozoites can infect liver cells within 10 minutes after mosquito bite

Each infected liver cell can release 10-30,000 merozoites

First cycle in liver (7-14 days)

Disease only commences once the parasite replicates in RBCs

Gametocytes come out of some RBCs (Allows for sexual mixing of malaria species when picked up by new mosquito)

\subsection*{Merozoite Form}
\begin{enumerate}
    \item many rounds of replication in RBCs
    \item Can infect as many as 60\% of RBCs
    \item Each time merozoites are released, they cause systemic inflammation and fever
    \subitem Causes cyclical fevers with the release of parasites from RBCs every 48-72 hours (Usually around midnight)
    \subsubitem Cytokines from lysed RBCs trigger fever
    \item Can cause severe anemia from loss of RBCs
\end{enumerate}

\subsection*{Clinical Aspects}
Diagnosis:
\begin{enumerate}
    \item Easy to diagnose by examining blood smear under microscope
    \item Cyclical spiking fever
    \item P. falciparum can be distinguished from other species in blood smears
\end{enumerate}

Mortality:
\begin{enumerate}
    \item In endemic areas almost 100\% of children have yearly symptomatic malaria
    \subitem Only 1-2\% have severe complications (cerebral malaria)
    \item P. falciparum causes most deaths from malaria brain infections where cerebral malaria has a mortality of 20\%
    \item Infection triggers inflammation which can result in neurological damage or death
\end{enumerate}

\subsection*{Immune Response}
Innate Immunity:

Parasites eliminated by phagocytosis and C activation outside cells (doesn't work very well)

Adaptive immunity
\begin{enumerate}
    \item Complicated immune response to Malaria with different antigens expressed by different morphological forms in liver and RBCs
    \item Abs to sporozoites before liver and CTL against liver infected cells
    \item T helper cells and B cells provide antibody triggered lysis to kill infected RBCs
\end{enumerate}

\subsection*{Why is Malaria So Successful?}
\begin{enumerate}
    \item Well-adapted for huamans and no animal reservoir
    \item Evolved for transmission only in Anopheles mosquito
    \item Gametocytes represent sexual stage where they can infect mosquitos
    \item Very good at evading immune response
\end{enumerate}

\subsection*{Partial Immunity}
In endemic areas, many people get malaria regularly and therefore reduce symptoms (does not prevent transmission)

\subsection*{Immune Evasion}
Infected RBC have ``knobs'' (surface antigens) which helps it attatch (avoid liver and spleen)

\tab \indicates Variant Surface Antigens (VSAs) or Plasmodium falciparum erythrocyte membrane protein 1 (Pfemp1) antigens encoded by 60 genes

Best antibody targets for immune system

VSAs provide binding of infected RBCs to the vascular endothelium (cells lining blood vessels)

Ability of P. falciparum to change the expression of VSAs is an important form of antigenic variation (different from influenza or SARS which mutate) which makes it difficult to develop more than partial immunity

\subsection*{Anti-Malarial Drugs}

Original drug was chloroquine (quinine compund similar to what is found in tonic water)

Most other drogs target heme (inhibiting heme-polymerization), leading to heme-toxicity to the parasites

Plasmodium parasites have resistance to all malaria drugs

\subsection*{Non-Medical Malaria Prevention}

Mosquito control:
\begin{enumerate}
    \item Pyrethoid-insecticide (permethrin) treated bednets
    \item 22\% decrease in malaria deaths in children \(<\)12 months
\end{enumerate}

\subsection*{Natural Resistance to Malaria}

HbS (Sickle Cell Hemoglobin):
\begin{enumerate}
    \item Individuals who are hederozygous have some protective advantage
    \item Mechanism is unclear but it is possible that infected cells tend to sickle under low oxygen tension conditions (Only a few percent of cells are sickled in uninfected heterozygous individuals)
    \item Sickle cell trait provides 60\% protection against overall mortality primarily during 2-16 months of age
    \item Frequency of sickle cell carriers (heterozygous) can be up to 25\% in endemic areas
\end{enumerate}

Blood Type:
\begin{enumerate}
    \item Uninfected O type RBC are less likely to ``rosette'' (form clumps of RBCs) to prevent killing
\end{enumerate}

\subsection*{Recombinant Vaccine against Pre-liver Stage}
RTS,S is composed of a P. falciparum circumsporozoite protein fused with the hepatitis B virus surface antigen combined with a GlaxoSmithKline proprietary adjuvant

\begin{enumerate}
    \item R: central repeat region of Plasmodium falciparum circumsporozoite
    \item T: T-cell epitopes of CSP
    \item S: hepatitis B surface antigen (HBsAg)
\end{enumerate}

Recommended by WHO and GAVI

Requires 4 doses for protection

Most significant was 30\% reduction in cases of severe malaria

Vaccine composed of 18 B cell epitopes and 5 T cell epitopes

\end{document}