\documentclass{notes}
\graphicspath{{../Images/}}

\fancyhead[l]{Jonathan Cheng}
\fancyhead[c]{MCB 55}
\fancyhead[r]{September 13, 2024}
\fancyfoot[c]{Page \thepage\ of \pageref{LastPage}}

\begin{document}
\section*{Antibiotics}
\subsection*{What do Antibiotics Target?}
In bacteria, the target unique and essential features required for growth

\begin{enumerate}
    \item Cell wall - Penicillin
    \item Protein synthesis - Streptomycin
    \item RNA synthesis - Rifampin
    \item DNA replication - Ciprofloxacin
    \item Metabolism - Sulfa Drugs
\end{enumerate}

\subsection*{Antibiotic Use}
Antibiotic use can increase the chance of getting certain infections

Clostridium difficile is normally non-threatening because of many bacteria in colon

If antibiotics are taken, they are the only bacteria left because they are naturally antibiotic-resistant

C. difficile infection can be a life-threatening inflammation of the colon

Antibiotics are \textbf{not} very selective

Normal microbiota functions to keep pathogens at bay (metabolic competition)

\tab \indicates Antibiotics kill many beneficial organisms as a side effect

\subsection*{How Did We Get Here?}
\begin{enumerate}
    \item Selection for antibiotic resistant bacteria
    \item Spread of antibiotic-resistant bacteria
    \item Spread of antibiotic-resistant genes between different bacteria
\end{enumerate}

\subsubsection*{Ways for Bacteria to Become Antibiotic Resistant}
\begin{enumerate}
    \item Antibiotic inactivation / Degradation
    \item Efflux (Removal of antibiotic) (Opposite of influx)
    \item Decreased permeability (Antibiotics cannot cross cell wall)
    \item Target Modification (Nothing for antibiotic to attatch to)
\end{enumerate}

Bacteria can evolve new traits very quickly

\tab \indicates Penicillin resistance was discovered \underline{before} clinical use

\subsubsection*{Selection for Antibiotic Resistant Bacteria}
Misuse (not following full treatment) / overuse of antibiotics \textbf{vs.} extensive use of antibiotics in agriculture

\subsubsection*{Selection and Spread of Bacteria}
Antibiotics ``weed out'' succeptible bacteria

\subsubsection*{Spread of Antibiotic-Resistant Genes}
Horizontal Gene Transfer - Conjugation

\tab \indicates Mobile Plasmid (Circular DNA) duplicated and transferred between bacteria

Does not require the same species

\subsection*{E-S-K-A-P-E Pathogens}

Most common nosocomial (hospital acquired) infections

High rates of antibiotic resistance

\begin{enumerate}
    \item Enterococcus faecium
    \item Staphylococcus aureus
    \item Klebsiella pneumoniae
    \item Acinetobacter baumannii
    \item Pseudomonas aeruginosa
    \item Enterobacter species
\end{enumerate}

CDC estimates ESKAPE pathogens cause over 2 milion illnesses and 23,000 deaths per year in the US

\subsubsection*{Healthcare-Associated Infections}
Staphylococcus Aureus

\begin{enumerate}
    \item Bacteremia / Sepsis - Bacteria in the bloodstream
    \item Pneumonia - Usually in patients with underlying lung disease or with mechanical ventilators
    \item Endocarditis (infection of heart valves) - Can lead to heart failure of stroke
    \item Osteomyelitis (bone infection) - Can happen after bacteremia or surgery / injury
\end{enumerate}

Major causes of nosocomial infections
\begin{enumerate}
    \item Catheter
    \item Surgical site
    \item Cancer treatments
    \item Hemodialysis
    \item Ventilators
\end{enumerate}

Before antibiotics S. aureus has a mortality rate of 82\%

Penicillin (1942) decreased deaths, but by 1950 almost 25\% were resistant to penicillin

Methicillin and oxacillin developed to overcome resistance (failed within a year)

MRSA = Methicillin Resistant Staphylococcus Aureus

\tab \indicates Responsible for 30\% of hospital-acquired infections

\tab \indicates Around 30\% are in the nose

\tab \indicates 43-58\% of S. aureus in pneumonia and surgical sites were MRSA

\end{document}