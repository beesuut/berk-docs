\documentclass{notes}
\graphicspath{{../Images/}}

\fancyhead[l]{Jonathan Cheng}
\fancyhead[c]{MCB 55}
\fancyhead[r]{September 11, 2024}
\fancyfoot[c]{Page \thepage\ of \pageref{LastPage}}

\begin{document}

\section*{Staphylococcus Aureus Bacteria}
S. aureus are:
\begin{enumerate}
    \item Cocci shaped (round) bacteria
    \item Non-motile (no flagella)
    \item Commonly found as harmless commensal bacteria on skin
    \item Replicate extracellularly
    \item Use fast replication + toxins to survive in host
    \item Regarded as an extracellular bacterium
    \subitem However it can invade and survive within human cells
    \subsubitem Evades immune system, antibiotic treatment, and allows bacterial proliferation
    \subsubitem Host cell exit is associated with cell death, tissue damage, and spread of infection
    \subsubitem Fluid inflammation helps spread bacteria
\end{enumerate}

S. aureus can inhabit nasal passages of up to 40\% of humans without causing disease

Is an opportunistic pathogen

\tab \indicates When able to invade tissues (get under skin, into blood), they can cause disease

\subsection*{Toxins}
Toxins are virulence factors released by bacteria to provide better survival in the host
Most toxins cause damage to the host or access to ther tissues in host or immune evasion or better transmission
Common toxins are:
\begin{enumerate}
    \item Hemolysin (Red Blood Cells)
    \item Leukotoxin (White Blood Cells(Neutrophils))
    \item Exfoliative Toxins (Removes Skin for access)
    \item Enterotoxins (Food poisioning)
    \item Toxic Shock Syndrome Toxin-1 (TSST-1)
\end{enumerate}

\subsection*{Diseases}
\subsubsection*{Non-Healthcare Associated Diseases}
Local skin infection
\begin{enumerate}
    \item Pimples
    \item Boils
\end{enumerate}

Toxin-mediated disease
\begin{enumerate}
    \item Impetigo
    \subitem Exfoliative
    \subitem Primarily in children
    \subitem Treated with antibiotics
    \item Staphylococcal Scalded Skin Syndrome
    \subitem Exfoliative
    \subitem Primarily in children
    \subitem Treated with antibiotics
    \subitem Causes scarring but usually not death
    \item Staphylococcus toxin-mediated food poisoning
    \subitem Staphylococcal Enterotoxins (SEA, SEB, SEC)
    \subitem Bacteria can be killed but toxins are heat stable
    \subitem No fever but severe nausea and vomiting
    \item Toxic Shock syndrome
    \subitem menstrual
    \subitem nonmenstrual
\end{enumerate}

\subsubsection*{Healthcare Associated Diseases}
\begin{enumerate}
    \item Bacteremia or depsis when bacteria spreads to bloodstream
    \item Pneumonia usually found in patients with lung disease or on mechanical ventilators
    \item Endocarditis (infection of heart valves) whcih can lead to heart failure or stroke
    \item Osteomyelitis (bone infection) can happen after bateremia or surgery/injury
\end{enumerate}

\subsection*{Toxic Shock Syndrome}
\subsubsection*{Background}
First used in 1978 to describe staph outbreak in children

Bacteria could not be isolated from blood, indicating a toxic might be involved

January 1980 - Epidemiologists in Wisconsin and Minnesota reported appearance of TSS mostly in menstruating women

\tab \indicates Acute fever, vomiting, rash, and high number of deaths

September 1980 - Users of Rely tampons were at greater rist for TSS

Tampons develpied in 1936, 90\% sold by Tampax

No regulations, but widespread usage by 1960

\subsubsection*{Rely Tampons}
To compete with Tampax, other companies made "super absorbent" tampons with synthetics, not cotton

\tab \indicates Rely tampons were made with polyester and carboxymethyl cellulose

Excessive absorbency led to altered ``microbial ecosystem''

Viscosity of vaginal fluids increased

Increased bacterial growth

New strain secreting TSST-1 now found in about 20\% of isolates

\subsubsection*{Menstrual TSS}
Starts within 2 days of beginning or end of menses

Associated with high absorbency tampons

\subsubsection*{Nonmenstrual TSS}
Caused by colonization of various sites (lung, skin, surgery)

Wounds soun't seem inflamed

Can be caused by TSST-1 (\textless 50\%) or other toxins / superantigens

Proper treatment results in mortality \textless 5\%

\subsection*{Superantigens}
Toxins respnsible for TSS and Staph mediated food poisoning are superantigens

Cause non-antigen specific mediated inflammation

\tab \indicates Superantigens ``glue'' the TCR to the MHC regardless of peptide

Less than 0.01\% of T-cells normally respond to a given pathogen

\tab \indicates T cells in immune response don't help with removing the bacteria / virus

Process:
\begin{enumerate}
    \item Activavte lots of T cells
    \item Make lots of cytokines
    \item Activate lots of macrophages
    \item Make more cytokines
    \item Create inflammation

\end{enumerate}


\end{document}