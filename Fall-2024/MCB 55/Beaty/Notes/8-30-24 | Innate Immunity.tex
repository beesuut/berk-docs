\documentclass{notes}
\graphicspath{{../Images/}}

\fancyhead[l]{Jonathan Cheng}
\fancyhead[c]{MCB 55}
\fancyhead[r]{August 30, 2024}
\fancyfoot[c]{Page \thepage\ of \pageref{LastPage}}

\begin{document}

\section{Innate Immunity}
The point of the immune system is to defend against pathogens

\hspace*{10px}A primary role of innate immunity is to activate acute inflammation

\hspace*{10px} \(\implies\) to understand immunity, you need to appreciate something about many different pathogens


\subsection{Size Matters}
Pathogens range in size
\begin{enumerate}
    \item Virus: \(0.01 \si{\mu m}\)
    \item Bacteria: \(1-10 \si[]{\mu m}\)
    \item Protozoa (Ex: Sleeping Sickness): \(10-100 \si{\mu m}\)
    \item Worms: \(1-3 \si{cm}\)
\end{enumerate}
\hspace*{10px} \(\implies\) Immune system doesn't work well on worms

\subsection{White Blood Cells}

White blood cells circulate through blood vessels and the lymph system

\hspace*{10px} Ex: Spleen

All white blood cells are immune cells (Surveil for signs of infection)

Initiation of immune responses often begin in lymph nodes nearest to site of infection

\hspace*{10px} Many lining intestines and none below knees


\subsection{Innate Immune Cells}

Ex: Macrophage, Neutrophil, Dendritic Cell

\subsubsection*{Key Aspects}
\begin{enumerate}
    \item All phagocytic and can kill bacteria and virus by engulfing them
    \item Release mediators which directly kill pathogens
    \item Make cytokines that can trigger more inflammation and attract other cells
    \item Macrophage and Dentritic Cells activate active immunity
\end{enumerate}
\subsubsection*{Differentiation}
\begin{enumerate}
    \item Macrophage
          \subitem Found is virtually every tissue
          \subsubitem Can engulf and kill bacteria
          \subitem Granules
          \subitem Generally come later than neutrophils
          \subsubitem Sense and clean old (Phagocytose senescent) / infected / dead (Apoptotic) cells
          \subitem Orchestrate inflammation
          \subsubitem Cytokines, proteases, and nucleases
    \item Neutrophil (Polymorphonuclear Cells (PMNs) or Granulocytes)
          \subitem Granules
          \subsubitem Contain enzymes that kill bacteria
          \subsubitem Contain reactive oxygen species (ROS) that kill bacteria
          \subsubitem Can also cause tissue damage \(\implies\) dangerous to release
          \subitem Odd shaped nucleus
          \subsubitem More nuclear membrane \(\rightarrow\) more activation proteins
          \subitem Consume and then commit suicide
          \subsubitem Usually short-lived (1-3 days once activated)
          \subsubitem Replenished from bone marrow
\end{enumerate}

\subsubsection*{Phagocytosis}
\begin{enumerate}
    \item The engulfment of particulate material within a membrane-bound intracellular compartment called a phagosome
    \item Material is degraded by enzymes found in lysosomes
    \item Mechanism of endocytosis is critical for ``killing'' bacteria and viruses
          \subitem Lysis: Killing by dissolving cell membrane
          \subitem Lysosomes at a lower PH (used as a killing method)
\end{enumerate}

\subsection{Inflammation}
Classic example: Wooden splinter (Breaks skin and invites pathogens)

\subsubsection*{Effects of Acute Inflammation}
\begin{enumerate}
    \item Brings innate immune cells to an infection site to phagocytize (kill) pathogens
    \item Brings fluid to ``cleanse'' site
    \item Brings immune molecules such as antibodies
    \item Can increase ``signals'' to activate adaptive immune system
    \item Unfortunately also damages uninfected cells
\end{enumerate}

\subsubsection*{Symptoms of Inflammation}
\begin{enumerate}
    \item Redness
    \item Pain
    \item Swelling
    \item Heat
\end{enumerate}

Excessive acute inflammation can lead to severe disease such as pneumonia

\hspace*{10px}Chronic inflammation is part of many diseases (autoimmunity, cancer, heart disease, Alzheimer's)

\subsection{How do Innate Immune Cells Recognize Pathogens?}
\begin{enumerate}
    \item Pathogen-associated molecular patterns (PAMPs)
          \subitem Cells have Toll-like receptors (TLR) that bind to PAMPs and send signals that start immune responses
          \subsubitem Distinct surface structures have become PAMPs
          \subsubitem Bacteria (prokaryotes) are structurally distinct from eukaryotes
    \item Ex: Lipopolysaccharide (LPS): Part of the outer membrane of gram-negative bacteria
    \item TLRs release cytokines that act as `alert signals' to activate inflammation
          \subitem Tumor necrosis factor (TNF) is an example of a cytokine
\end{enumerate}

\subsection{PAMPs}
The best PAMPs are those not found in our cells

PAMPs on pathogens cause activavtion of TLRs

\hspace*{10px} \(\rightarrow\) Produces cytokines/chemokines that activate inflammation
and act as signals to activate adaptive immunity

\hspace*{10px} \(\rightarrow\) Amount of inflammation is dependent on the amount of bacteria

Successful immune response involves balanced cytokine response

Too many cytokines \(\implies\) `shock' or death

Too few cytokines \(\implies\) failure to contain infection

\subsubsection*{Viral DNA and RNA can be PAMPs}
 Why don't we recognize our host cell's DNA/RNA?

\hspace*{10px} \(\rightarrow\) Usually found in a place or form different from normal

Ex:
\begin{enumerate}
    \item Viral DNA in the cytosol is a PAMP. Host cell DNA is normally only in the nucleus
    \item Double-stranded viral RNA is a PAMP. Host cell self RNA is normally single stranded
\end{enumerate}

\end{document}