\documentclass{notes}
\graphicspath{{../Images/}}

\fancyhead[l]{Jonathan Cheng}
\fancyhead[c]{MCB 55}
\fancyhead[r]{September 16, 2024}
\fancyfoot[c]{Page \thepage\ of \pageref{LastPage}}

\begin{document}

\section*{Tuberculosis}
TB (disease) was called ``consumption'' because patients appared to be consumed from within

Tuberculosis is called Mycobacterium tuberculosis (Mtb)

\tab \indicates MTB are bacilli-shaped bacteria

Discovered in 1882 by Robert Koch

\tab \indicates Previously assumed to be an inherited disease before bacterium was discovered

1.6 million deaths to TB per year (more than any other infectious disease until COVID-19)

\tab \indicates Has been causing a COVID-scale pandemic every year for over 200 years

\tab \indicates 90\% of deaths occur in the developing world

\subsection*{Challenges of Mtb Research}
Must be handled at Biosafety level 3 out of 4 BSL levels

\tab \indicates Control over airflow / sanitation (expensive and difficult)

Mtb grows very slowly

\begin{enumerate}
    \item Divides every 20 hours
    \item Takes 2-3 weeks to form a colony
    \item Hard to perform rapid antibiotic susceptibility testing
    \item Hard to treat (antibiotics preferentially kill dividing bacteria)
\end{enumerate}

\subsection*{Symptoms}
\begin{enumerate}
    \item Bloody cough
    \item Weight loss (wasting)
    \item Fever
    \item Nausea
\end{enumerate}
Slow progression but often leads to death

Has been causing disease in humans for at least 50,000 years but accelerated by industrual revolution

\subsection*{Characteristics of Mtb}
\begin{enumerate}
    \item Humans are the only natural reservoir of Mtb
    \subitem Mice can be infected, but not naturally
    \item Mtb is an \textbf{\underline{intracellular} bacterial pathogen}: can ONLY replicate inside human cells
    \item Has a thick waxy coat which provides resistance to immune response and antibiotics
    \subitem Extra layers of mycolic acids (long chains of hydrocarbons) serving as a lipid barrier
    \subitem Resistant to phagocytosis / poor antibody binding
\end{enumerate}

\subsection*{Transmission}
After inhalation, Mtb initially replicates in alveolar macrophages

\tab \indicates Infectious dose can be as low as 1 bacterium

\tab \indicates Waxy coat / Disruption of phagosome maturation

Early ``5 minute'' phagosome doesn't fully acidify into a lysosome, so the bacteria are not killed

\subsection*{TB Life Cycle}

\begin{enumerate}
    \item Aerosol transmitted; low infection dose (\(\sim\)1-10 CFU)
    \item Initially infects anveolar macrophages
    \item Very heterogenous and complex outcomes
    \item Active disease associated with uncontrolled lung inflammation in about 10\% of individuals
    \subitem 80\% of infections result in latent TB (No symptoms or transmission)
    \subsubitem Few can eradicate the bacteria
    \subsubitem Bacteria trapped in granuloma but cannot be eradicated
    \subsubitem Can be ``converted'' to active TB in \(\sim\) 10\% of population
    \subsubitem Keeps TB in population
\end{enumerate}

\subsection*{Granulomas}
Formerly called tubercules, granulomas are organized aggregates of immune cells and bacteria

\tab \indicates Helps contain but not (necessarily) eliminate Mtb

A ``hallmark'' of Mtb infection

\subsection*{Vaccination}
Baccile Calmette-Guerin (BCG) is a live-attenuated vaccine from cows

Given to many kids in developing countries and protects against TB meningitis (infection of nervous system)

Does not protect against pulmonary TB in adults (major form in adults)

\subsection*{Latent to Active TB}

Normally:
\begin{enumerate}
    \item 5\% chance of conversion within a year
    \item 5-15\% change in remaining lifetime
\end{enumerate}

Someone with AIDS:
\begin{enumerate}
    \item 8\% chance per year
    \item 80\% chance within 10 years
\end{enumerate}

Patients on anti-TNF (Humira, Remicade, etc.) also at increase risk

\subsection*{Methods of Diagnosis}
Chest X-Rays: Characterized by cavitary lesions

\tab \indicates Imprecise, and latent TB may not show up at all

Smear Test: Patients cough up sputum and Mtb bacteria in sputum can be visualized with acid-fast (Ziehl-Neelsen) stain

\tab \indicates Inexpensive, fast, and specific

\tab \indicates Few false positives, but many false negatives (Needs \(>\) 10,000 bacilli/mL)

\tab \indicates cannot detect latent TB

Culture Test: Grown from sputum sample

\tab \indicates More sensitive than smear test but slower and more difficult

\tab \indicates Solid (3-4 weeks); Liquid (2 weeks)

Latent TB cannot be detected directly

\tab \indicates Diagnosed indirectly by looking for immune response to Mtb

\begin{enumerate}
    \item Tuberculin skin test (TST): Immune reaction to TB antigens in skin
    \item Interferon gamma release assay (IGRA): Measures an immune reaction to TB antigens in a test tube
\end{enumerate}

\subsubsection*{TST}
Purified protein derivative (PPD) consists of ``purified'' Mtb antigens

Injected intradermally and then observed 48-72 hours later for a bump forming at site

Anyone given BCG vaccine will give a positive test even if uninfected

\subsubsection*{IGRA}
Detects a cytokine called interferon gamma (produced by T cells in response to Mtb antigens)

Detects antigens unique to Mtb and not found in BCG

\subsection*{Antibiotics}

First Line Antibiotics
\begin{enumerate}
    \item Isoniazid (INH) \textbf{Memorize}
    \item Rifampin (RIF) \textbf{Memorize}
    \item Pyrazinamide (PZA)
    \item Ethambutol (EMB)
\end{enumerate}

Second Line Antibiotics
\begin{enumerate}
    \item Fluoroquinolones (e.g. ciprofloxacin)
    \item Injectables
\end{enumerate}

6 month first-line treatment for drug-sensitive TB

Multi drug resistant TB (MDR TB) resistant to INH and RIF (Requires 1-2 years of second line antibiotics)
\begin{enumerate}
    \item 4\% of new cases and 19\% of previously treated cases in 2016 were MDR TB
    \item 240,000 people died from MDR TB
    \item 6\% of people with MDR TB have XDR TB
\end{enumerate}

Extensively drug resistant TB (XDR TB) resistant to INH, RIF, and some second line drugs (often incurable, surgery sometimes required)

\subsection*{HIV-AIDS and TB}
\begin{enumerate}
    \item 
\end{enumerate}
\end{document}