\documentclass{notes}
\graphicspath{{../Images/}}

\fancyhead[l]{Jonathan Cheng}
\fancyhead[c]{MCB 55}
\fancyhead[r]{August 28, 2024}
\fancyfoot[c]{Page \thepage\ of \pageref{LastPage}}

\begin{document}

\section{Emergent Infections Diseases}
Mostly zoonotic (from animals), however the ``Big Killers'' can be very different

\begin{enumerate}
    \item Retrovirus (HIV-AIDS)
    \item Virus (SARS)
    \item Bacterium \{Prokaryotic\} (Tuberculosis)
    \item Protozoan Parasite \{Eukaryotic\} (Malaria)
\end{enumerate}

\section{Infection Vs. Diseases}
Pathogens don't `want' to cause sickness

Only cause disease if it helps transmission

\hspace*{10px} Ex: Dengue Virus spread by mosquitos \(\rightarrow\) Asymptomatic

\subsection{Levels of Symptoms}
\begin{enumerate}
    \item Death
    \item Severe disease
    \item Symptomatic
    \item Asymptomatic
\end{enumerate}

\section{Epidemiology}
What causes disease and how are they spread?

Miasma Theory (bad air \{mala-aria\}) \(\rightarrow\) Germ Theory (1800s)

\subsection{Terminology}
\begin{enumerate}
    \item Endemic: An infection that is continually present in a population or geographic arrangement
    \item Epidemic: An outbreak of infectious disease above the normal level of infections that then subsides
          \begin{enumerate}
              \item Outbreak: A minor increase in infections
          \end{enumerate}
    \item Pandemic: Infections that spread over more than 3 continents
    \item Innate Immune System
          \subitem Limits the spread of pathogens during infection
          \subitem Activates adaptive immune system
    \item Adaptive Immune System
          \subitem Fully eliminates pathogens
          \subitem Prepares for a second infection with same pathogen
\end{enumerate}


\section{Cell Biology Review}

\subsubsection*{DNA - Genetic code}

\begin{enumerate}
    \item DNA is transcribed to single-stranded RNA
          \subitem RNA is translated to proteins
          \subitem Viruses can have both DNA or RNA
\end{enumerate}

\subsubsection*{Proteases (enzymes) degrade proteins into peptides}

\begin{enumerate}
    \item Peptides act as signals for immune response
    \item Peptides degrade into amino acid
\end{enumerate}

\subsubsection*{Recombinant DNA}

\begin{enumerate}
    \item DNA that has been spliced together
    \item For use in research or vaccines
\end{enumerate}

\subsection{Animal Cell Structure}

\subsubsection*{Plasma Membrane}

\begin{enumerate}
    \item Separates interior and exterior of cell
    \item Made of phospholipid bilayer
\end{enumerate}

\subsubsection*{Cytoplasm}

\begin{enumerate}
    \item Gelatinous fluid filling a cell
    \item Made primarily of water and salts
\end{enumerate}

\subsubsection*{Ribosome}

\begin{enumerate}
    \item Site of protein synthesis
    \item Reads messenger RNA (mRNA) and translates code into a string of amino acids
    \item Made of RNA and protein
\end{enumerate}

\subsubsection*{Mitochondria}

\begin{enumerate}
    \item Source of chemical energy in cell
          \subitem Forms adenosine triphosphate (ATP)
\end{enumerate}

\subsubsection*{Golgi Body / Golgi Apparatus}

\begin{enumerate}
    \item Vesicles and folded membranes involved in secretion and transport
    \item Receives proteins from ER
    \item Delivers proteins and lipids
\end{enumerate}

\subsubsection*{Endosome (Membrane-bound vesicle)}

\begin{enumerate}
    \item Sort and transport components within the cell
    \item Hold things inside membranes
\end{enumerate}

\subsubsection*{Lysosome}

\begin{enumerate}
    \item Breaks down excess or used cell parts
    \item Example of an endosome
\end{enumerate}

\subsubsection*{Endoplasmic Reticulum (ER)}

\begin{enumerate}
    \item Smooth
          \subitem Produces phospholipids
          \subitem Useful for cell membranes and metabolism of carbohydrates
          Transports products of rough ER to golgi apparatus
    \item Rough
          \subitem Covered in ribosomes
          \subitem Prominent part of protein synthesis
\end{enumerate}

\subsubsection*{Centriole}

\begin{enumerate}
    \item Organelle in cytoplasm which aid in cell division
    \item Determines the position of the nucleus and arrangement of the cell
\end{enumerate}

\subsubsection*{Nuclear Envelope}

\begin{enumerate}
    \item Separates nucleus from cytoplasm
    \item 2 lipid bilayer membranes
\end{enumerate}

\subsubsection*{Nucleus}

\begin{enumerate}
    \item Contains genetic material
    \item Stores and replicates DNA
    \item Uses DNA transciption to form RNA
    \item Defining characteristic of eukaryotic cells
\end{enumerate}

\subsubsection*{Nucleolus}

\begin{enumerate}
    \item Within cell nucleus
    \item Transcribe ribosomal RNA and assemble ribosomes (ribosome biogenesis)
\end{enumerate}


\subsection{Processes and Properties}

\subsubsection*{Phagocytosis (Cell Eating Process)}
Form of endocytosis that

\begin{enumerate}
    \item uses pseudopodia to engulf particles and results in the formation of phagosomes
    \item which can fuse with lysosomes containing enzymes
    \item to break down into smaller pieces
\end{enumerate}

Carried out by specialized immune cells in humans

\hspace*{10px} Ex: Neutrophils, macrophages, and dentritic cells

\subsubsection*{Receptors and Ligands}

\begin{enumerate}
    \item Receptors on surface of cells bind to soluble or (another) cell-surface ligand
          \subitem Activates an action for the cell
    \item Signal transduction - signal sent through cell to trigger response
          \subitem Often new protein production or other change
\end{enumerate}

\subsubsection*{Cytokines}

\begin{enumerate}
    \item Molecular messengers that control and regulate the immune system
          \subitem ``Immune Hormones''
    \item Secreted ligands recognized by receptors on cell surfaces
          \subitem Autocrine action
    \item Released by one cell to affect another cell
          \subitem Paracrine action (nearby cell)
          \subitem Endocrine action (distant cell through blood)
\end{enumerate}


\end{document}