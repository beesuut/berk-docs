\documentclass{notes}
\graphicspath{{../Images/}}

\fancyhead[l]{Jonathan Cheng}
\fancyhead[c]{MCB 55}
\fancyhead[r]{September 18, 2024}
\fancyfoot[c]{Page \thepage\ of \pageref{LastPage}}

\begin{document}

\section*{Escherichia Coli}
Gram negative bacillus (rod)

Many flagella

Grows very fast (20 minute doubling time)

\subsection*{Genetic Diversity}
Every E. coli strain must have 2,167 core genes (Most viruses have around 20 total)

4,271 total genes on average

10,131 possible genes (Most strains are benign)

Acquisition of mobile genetic elements can change E. coli from a commensal to a highly adapted and deadly pathogen

\subsection*{8 ``Pathovars'' of Pathogenic E. Coli}
Serovar: Distinct variation of bacteria

Pathovar: Serovar for pathogens

\(\sim\)400 million infections

Types of E. Coli:
\begin{enumerate}
    \item Enterohaemorrhagic (EHEC)
    \item Enteropathogenic (EPEC)
    \item Uropathogenic (UPEC)
    \item Diffusely Adherent (DAEC)
    \item Enteroinvasive (EIEC)
    \item Enteroaggregative (EAEC)
    \item Neonatalmeningitis (NMEC)
\end{enumerate}

\subsection*{EPEC vs. EHEC / STEC}
EPEC:
\begin{enumerate}
    \item Causes childhood diarrhea
    \item Transmits human to human
    \item No Shiga toxin
\end{enumerate}

EHEC / STEC
\begin{enumerate}
    \item Food/water-borne pathogen in industrialized countries
    \item Zoonotic from cattle or other animals
    \item Secreted Shiga toxin can lead to Hemolytic Uremic Syndrome (HUS)
    \subitem Lysed cells can clog kidneys and lead to kidney failure
    \item 265,000 illneses each year in the US
    \item 3,600 hospitalizations
    \item 30 deaths
\end{enumerate}

Both have same mechanism of intestinal colonization

\subsubsection*{How do bacteria not get washed out of intestine?}
Attatch via pedestals

EHEC / EPEC induce pedestal formation to stay attatched

\begin{enumerate}
    \item Inject E. coli receptor protein into epithelial cells
    \item E. coli protein in epithelial cells binds to the bacteria enabling attachment
\end{enumerate}

\subsection*{Serotyping by E. Coli Antigens}
Surface antigens include polysaccharide side chains (O antigen), capsular antigen (K), and flagellar protein (H)

200 O antigens, 80 K antigens, and 56 H antigens

\tab \indicates Flagella is 1 PAMP, but 56 different possible antigens

\section*{EHEC / STEC}
EHEC (Shiga toxin producing E. coli)

Found on contaminated beef

Several serotypes in EHEC frequently associated with human disease

\begin{enumerate}
    \item \textbf{O157:H7}
    \item O26:H11
    \item O91:H21
    \item O111:H8
    \item O157:H7
\end{enumerate}

How does Shiga toxin work?

\begin{enumerate}
    \item Shiga toxin (Stx) genes are found in pathogenic E. coli and Shigella dysenteria
    \item Forms a pentamer of B subunits that bind and enter host cells, allowing a single A subunit to enter the cell
    \item The A subunit of the toxin injures the eukkaryotic ribosome and inhibits protein synthesis in target cells and can kill cells
    \item Shiga toxin can attack epithelial cells, endothelial cells, and immune cells
    \item Shiga toxin can attack cells in intestine (colitis / diarrhea) and kidney / endothelial cells in kidney (HUS)
    \item Antibodies to Stx are protective against severe disease
\end{enumerate}


\subsection*{Outbreaks}
Jack in the Box: 1992
\begin{enumerate}
    \item Over 600 people infected in 6 states
    \subitem Mostly children
    \subitem 4 deaths
    \subitem 50 cases of kidney failure from HUS
    \item Cause: Knowingly undercooking burgers
\end{enumerate}

2 class action suits and USDA began testing all ground beef in 1994

\(\sim\)100,000 EHEC infections each year in the US

\subsubsection*{How Did EHEC Acquire Shiga Toxin?}

\begin{enumerate}
    \item Virus infects and kills shigella
    \item Virus picks up Stx gene piece
    \item Virus infects commensal E. coli
    \item Stx phage integrates into E. coli genome
\end{enumerate}

\subsection*{Transmission in Animals}
Healthy cattle are the major reservoirs of E. coli O157:H7

Contaminated bovine products and crops are predominant sources for human infections

Animal transmission through fecal contamination of food or water

``Super-shedder'' cows: Colonized at rectum for long periods of time, shedding more than 95\% of E. coli in a herd

\subsection*{Human Transmission}

Undercooked or unpasteurized animal products
\begin{enumerate}
    \item Ground Beef
    \item Other Meats
    \item Milk, Cheese
\end{enumerate}

Foods contaminated with animal feces
\begin{enumerate}
    \item Fruits
    \item Vegetables
\end{enumerate}

Contaminated water
\begin{enumerate}
    \item Wells
    \item Swimming (lakes, streams)
\end{enumerate}

Contaminated soil
\begin{enumerate}
    \item Campgrounds
    \item Sites grazed by livestock
\end{enumerate}

\subsection*{Disease in Humans}
Hemorrhagic Colitis
\begin{enumerate}
    \item Bloody diarrhea
    \item Sever abdominal cramps
    \item Possibly fever, nausea, vomiting
    \item Many cases are self limiting and resolve in \(\sim\)1 week
\end{enumerate}

Hemolytic Uremic Syndrome (HUS)
\begin{enumerate}
    \item Children, elderly, immunocompromised
    \item Kidney failure, hemolytic anemia, thrombocytopenia (tiny blood clots clog capillaries)
\end{enumerate}

\subsection*{Treatment}
\begin{enumerate}
    \item Mainly supportive
    \item Antibiotics are usually avoided as they may not reduce symptoms, prevent complications, nor reduce shedding of bacteria
    \item May even increase risk of HUS
\end{enumerate}

\section*{Salmonella Enterica}
Comprises a number of subspecies, all of which are common sources of food poisoning

2 main serovars:
\begin{enumerate}
    \item S. Typhimurium
    \subitem Causes gastroenteritis
    \subitem Short-term infection of GI tract
    \subitem Broad range humans / animals
    \item S. Typhi
    \subitem Typhoid fever (Makes typhoid toxin)
    \subitem Life threatening systemic infection
    \subitem 3-5\% are carriers that shed at high levels
    \subitem Human-specific
\end{enumerate}

\end{document}