\documentclass{notes}
\graphicspath{{../Images/}}

\fancyhead[l]{Jonathan Cheng}
\fancyhead[c]{MCB 55}
\fancyhead[r]{September 9, 2024}
\fancyfoot[c]{Page \thepage\ of \pageref{LastPage}}

\begin{document}

\section*{Acquisition of Immunity}
\subsection*{Passive Immunity}
Passive immunization = Transfer of antibodies

\tab \indicates Antibodies can be transfered between people (and animals)

\tab \indicates Any immune protection lasts as long as the antibodies are present

Examples:
\begin{enumerate}
    \item Antitoxins (Antibodies from horses)
    \subitem Diptheria
    \subitem Tetanus
    \subitem Botulinum
    \item Antivenoms
    \item Human gammaglobulin (IgG from many plasma donors)
\end{enumerate}

\subsubsection*{Maternal}
\begin{enumerate}
    \item IgG through placenta
    \item IgA through breastmilk
\end{enumerate}

\subsection*{Active Immunization}
Vaccines are usually \textbf{active} immunizations

\tab \indicates Induce adaptive immune response similar to "natural infection"

\tab \indicates Establishes memory T and B cells

Most vaccines prevent disease but not infection
innoculations \indicates immunizations \indicates vaccines

immunization = any injection to illicit an immune response


\subsubsection*{Preventive Vaccines}
Given to naive individuals in order to provide protection from primary infection or prevent disease

\subsubsection*{Therapeutic Vaccines}
Given to infected individuals to prevent / reduce disease or stimulate anti-tumor response

\subsection*{Smallpox Eradication (Vaccinia)}
Why was eradication possible?

\begin{enumerate}
    \item No animal reservoir
    \item Lifelong immunity
    \item One serotype \(\rightarrow\) little antigenic variation
    \subitem No repeat infections
    \item Effective attenuated vaccine provided long-term immunity
\end{enumerate}

\subsection*{Routes for Vaccination}
Many vaccines are given intramuscular or subcutaneous

\tab \indicates Live polio virus and rotavirus vaccine given orally

Immunization site will influece where immune responses are elicited but most vaccines are tested in easy to administer route and assessed for protection

\subsubsection*{FluMist}
Intranasal vaccine for influenza

Replicates only in nasal cavity

\subsubsection*{Flu Vaccine}
100 million doses available each year

Usually trivalent (With 3 different viral strains)

Traditional approach: Identify ``new'' virulent strains, recombine, and grow in eggs

\subsubsection*{Subcutaneous or Intradermal}
Subcutaneous = under skin

Intradermal = between skin layers (can see bubble under skin)

Ex: Monkeypox

Why this method?

\tab \indicates Supposedly because of focus on skin

\tab \indicates Likely just how it was tested

\subsection*{Vaccines need Adjuvants}
Adjuvants added in order to enhance immunogenicity

\tab \indicates Activate macrophages and DCs to increase inflammation

Old adjuvants were inorganic salts such as aluminum hydroxide, aluminum phosphate, or calcium phosphate

\tab \indicates Newer adjuvants provide ligands to bind TLRs

\tab \indicates AS01-AS06 are oil in water emulsions with Lipid A, or Saponin (detergent), or CpG (DNA)

\tab \tab \indicates Contain PAMPS to activate inflammation via TLRs

\subsection*{Dealing with Antigenic Variation}
Some viruses have many different antigenic subtypes and high mutation rates

\section*{Types of Vaccines}

\subsection*{Attenuated Vaccines}
Made by growing pathogen in non-human cell culture until pathogen is less virulent

\tab \indicates Less virulent = Less pathogenic = Less disease-causing

Ex:
\begin{enumerate}
    \item Oral Polio Vaccine (OPV)
    \item Measles
    \item Mumps
    \item Rubella
    \item Varicella Zoster Virus (VZV)
\end{enumerate}

Advantages:
\begin{enumerate}
    \item Self-replicating (low dose)
    \item Real virus
    \subitem No adjuvant
    \subitem Authentic antigen presentation
    \item More effective at eliciting CTLs
\end{enumerate}

Disadvantages
\begin{enumerate}
    \item If vaccine replicates, it could infect other people and become virulent
    \item Does not deal with strain variability / antigenic variation
\end{enumerate}

\subsection*{Inactivated Vaccines}
Killed (inactivated) whole organism OR inactivated toxin

\tab \indicates Done by heat, chemicals, or irradiation

Ex:
\begin{enumerate}
    \item Influenza
    \item Hepatitis A
    \item Pertussis
    \item Salk inactivated polio vaccines (IPV)
    \item Tetanus
    \item Diptheria
\end{enumerate}

Advantages: No revirulence (more safe)

Disadvantages: No replication (poor antigen presentation)

\subsection*{Live Vector Vaccines}
Insert genes from pathogen into a well characterized vaccine vector

Ex: 
\begin{enumerate}
    \item Vaccinia
    \item Adenovirus
    \item Salmonella
    \item Vesicular stomatitus
\end{enumerate}

Advantages:
\begin{enumerate}
    \item Self replicating
    \item No adjuvant needed
\end{enumerate}

Disadvantages: Live vector could be an issue with potential pathogenesis

\subsection*{Recombinant Protein Vaccines}
Identify immunogenic proteins by its envelope or outer membrane

Advantages: Less expensive and very safe

Disadvantages: Need adjuvant and boosters (may not elicit long-term memory)

Ex: Human Papilloma Virus (HPV) Vaccine

\tab \indicates Mimics structure without containing any HPV viral DNA

\subsection*{Viral Spike Protein}
Key target of neutralizing antibodies

Moderna's SARS CoV-2 mRNA Vaccine
\begin{enumerate}
    \item mRNA for spike protein inside lipid nanoparticles
    \item Host cells "uptake" the spike and generate the protein
    \item Proteins get released from cell to generate Th antibody response
\end{enumerate}

Advantages:
\begin{enumerate}
    \item Fast production and ease for inserting new viral strain into same mRNA vector
    \item Mimics aspects of infection
    \subitem Gets into cells
    \subitem Makes proteins
    \subitem Activates TLRs
    \item Safer than most vaccine types
\end{enumerate}

Disadvantages:
\begin{enumerate}
    \item Must be encapsulated in lipids or sugars
    \item Must be frozen
    \item Not very stable once thawed
\end{enumerate}

Johnson and Johnson Adeno/Spike vaccine (66-72\% efficacy)
\begin{enumerate}
    \item Spike gene added to Adenovirus 26
    \item Adenoviruses are common viruses
\end{enumerate}

\subsection*{Issues in Vaccine Development}
Always tested first in young, healthy people

\subsubsection*{Efficacy}
Most common human vaccines are \textgreater 90\% protective

Influenza vaccine is 20-70\% protective

Different efficacy in adults, children, and immunocompromised individuals

\subsubsection*{Cost}
Most original attenuated vaccines were incredibly cheap

The WHO Big Six cost les than \$1 per person
\begin{enumerate}
    \item Diptheria toxoid
    \item Tetanus Toxoid
    \item Acellular pertussis (DTap)
    \item Polio
    \item Measles
    \item BCG
\end{enumerate}

New vaccines have high development costs

\tab \indicates Can take years to go down depending on volume of production

\end{document}