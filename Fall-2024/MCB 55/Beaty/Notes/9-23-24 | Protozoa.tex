\documentclass{notes}
\graphicspath{{../Images/}}

\fancyhead[l]{Jonathan Cheng}
\fancyhead[c]{MCB 55}
\fancyhead[r]{September 23, 2024}
\fancyfoot[c]{Page \thepage\ of \pageref{LastPage}}

\begin{document}

\section*{Parasites}
True parasites are eukaryotic organisms living in/on other eukaryotes

Protozoa are single-cellular parasites that can be intracellular or extracellular

Helminths (parasitic worms) are multicellular and can be up to meters long

\section*{Extracellular Protozoa}
Giardia Lamblia
\begin{enumerate}
    \item Lives in host intestines
    \item Inhibit phagocytosis
    \item Main adaptive response is IgA
    \item Zoonosis from beavers
    \item High asymptomatic rate
\end{enumerate}

Trypanosoma Brucei
\begin{enumerate}
    \item Blood dwelling trypanosomes
    \item Transmitted bewteen hosts via a tsetse fly vector
    \item Main immune response is IgG to VSGs
    \item Human trypanosomiasis is a zoonosis from cows
\end{enumerate}

\subsection*{Giardia Lamblia}
200 million symptomatic (diarrhea) infections

Up to 90\% of infection are asymptomatic (Strongly underdiagnosed and underreported)

Almost no mortality

Considered a zoonosis (Transfered between animals w/ only occasional human infections)

Parasites attatch to mucosal epithelium and can survive for weeks in host

Trophozoite form (w/ flagella) replicates in intestine

\tab \indicates A single trophozoite has 2 identical nuclei

Cyst form passed in feces

Beaver (maion animal reservoir in North America) but dogs and cats can be reservoirs

\subsection*{Trypanosoma Brucei}
Causes sleeping sickness (Inflammation in the brain puts people into a coma / causes brain damage)

Spread by tsetse fly bites

Live extracellularly in the bloodstream

Normally controlled by IgG but can evade IgG through antigenic variation

\subsubsection*{Immune Response to Trypanosomes}
Innate Immunity:
\begin{enumerate}
    \item Hard to eliminate by phagocytosis because of size
\end{enumerate}

Adaptive Immunity:
\begin{enumerate}
    \item IgG elimination of parasites by neutralization and ADCC
    \item Parasite can change the primary antigen through antigenic variation
    \subitem Variale surface glycoproteins (VSGs) help evade IgG
\end{enumerate}

\section*{Intracellular Protozoa}
\begin{enumerate}
    \item Malaria (Plasmodium) infects liver cells and RBCs
    \subitem Immune Response: CTLs to liver form, IgG to RBC form
    \item Protozoa that replicates in vacuole
    \subitem Leishmania chagasi infects macrophages
    \subsubitem Transmitted by sandflies
    \subsubitem Main immune response is Thelper cytokines activating macrophages
    \subitem Toxoplasma gondii lives in muscle and neurons as cysts
\end{enumerate}

\subsection*{Leishmania}
\begin{enumerate}
    \item Infects macrophages
    \item Sites of chronic inflammation
    \item Infected macrophages need to be activated by cytokines from Thelper cells to kill parasites in vesicles
    \subitem Infected macrophages also cause non-specific damage and immunopathology causing tissue damage
\end{enumerate}

Areas:
\begin{enumerate}
    \item Mucocutaneous: Mucosal Tract (Too much immune response (Destruction of tissue))
    \item Cutaneous: Skin (Balanced)
    \item Visceral: Body (Too little immune response)
\end{enumerate}

Disease manifestation:
\begin{enumerate}
    \item Different species cause different diseases
    \item Too little immune response is bad
    \item Too much immune response is worse
\end{enumerate}

\subsection*{Toxoplasma Gondii}
Felines are the ONLY definitive host (Mice are on intermediate host)

Humans are considered a dead-end host

Oocysts are ingested by mice or humans and form tissue cysts primarily in muscle or nerve cells

Humans can be infected by ingesting undercooked meat from animals that have cysts in tissue

Incredibly high prevalence of antibodies to toxoplasma antigens in humans


\end{document}