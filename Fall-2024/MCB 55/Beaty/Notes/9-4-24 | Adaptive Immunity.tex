\documentclass{notes}
\graphicspath{{../Images/}}

\fancyhead[l]{Jonathan Cheng}
\fancyhead[c]{MCB 55}
\fancyhead[r]{September 4, 2024}
\fancyfoot[c]{Page \thepage\ of \pageref{LastPage}}

\begin{document}

\section{Adaptive Immunity}

Adaptive immunity has specificity and memory

\hspace*{10px} \(\rightarrow\) Primary mechanism for differentiation self vs. foreign antigens

\hspace*{10px} \(\implies\) Must determine if a pathogen is foreign and harmful (pathogenic) vs foreign and non-harmful (non-pathogenic)

Innate immunity is always the same while the adaptive response can change

\subsection*{Primary Infection}

Takes a few days (3-7 days) to initiate
T and B cells against antigens must be activated and expanded (proliferate) in spleen or lymph nodes

\subsection*{Secondary Infection}

Adaptive response happens more quickly (1-3 days) because of adaptive immune memory cells
 
\subsection*{3 Tools of the Adaptive Immune Response}
\begin{enumerate}
    \item Antibodies made by B cells
    \item Cytokines made by T helper cells
    \item Killing of infected cells by cytotoxic T cells (CTLs)
\end{enumerate}

\subsection*{B Cell Mediated Immunity}
Antibodies produced by B-cells interact with pathogens and their toxic products in the blood or other extracellular spaces of the body

Main function of B cells is to produce antibodies

\hspace*{10px} \(\rightarrow\) B cells are the only cells capable of producing antibodies

\subsection*{T-Cell Mediated Immunity}
T cells only recognize antigens as a small peptide fragment (peptide) bound to an MHC molecule and displayed at the cell surface

\subsubsection*{Cytotoxic T Cells}
Kill virally-infected cells

\subsubsection*{T helper Cells}
Produce cytokines

\indicates Main regulators of adaptive immune responses

Needed for:
\begin{enumerate}
    \item B cells to produce antibodies
    \item Cytotoxic T cells in order to become maximally effective killers
    \item Macrophages to kill off pathogens inside
\end{enumerate}

\subsection*{Antibody Structure}
\subsubsection*{Antigen Binding Region}
One part of ab (Fab) binds specifically to antigens (pieces of pathogens)

The Fab part is highly variable (\(10^{11}\) different antibodies)

But the Fc (Effector function) part is the same across all humans

An individual B cell only produces a single type of antibody

Each antibody has the specific ability to bind only one antigen

\subsubsection*{Antibody Isotypes}
Determined by constant region (Fc)

Only a handful of different constant regions that can be found on antibodies and they result in different ``types'' or isotypes of antibodies

4 main antibody types
\begin{enumerate}
    \item IgM
    \item IgG
    \item IgA
    \item IgE
\end{enumerate}

\subsection*{Functions of Antibodies}
\begin{enumerate}
    \item Neutralization: Antibodies bind to antigens and block toxic function
    \item Opsonization: Fc of antibody binds to FcReceptor on phagocytic cell to provide enhanced uptake
    \item Complement Activation: Complement protein binds to Fc receptors
    \subitem Leads to increased uptake: opsonization
    \subitem Full complement pathway activation leads to cell lysis of pathogen or infected cell by pore formation
\end{enumerate}

Pathogens have many antigens \(\rightarrow\) Antigens have many epitopes

\hspace*{10px} \(\rightarrow\) Each antibody binds to a single epitope

\subsection*{Major Histocompatibility (MHC)}
Displays fragments of pathogens on the surface of host cells

Dendritic cells, B cells, and macrophages present antigens to T helper cells (professional antigen presenting cells)

DCs and macs engulf pathogens by phagocytosis, digesting it, and presenting it on cell surface

\subsection*{MHC Restriction}
T cell receptor only recognizes antigen in the ``context of'' self MHC

MHC restriction: the fact that a peptide can only be recognized by a given T cell if it is bound to a particular self-MHC

\end{document}