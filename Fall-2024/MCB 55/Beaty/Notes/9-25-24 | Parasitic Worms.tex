\documentclass{notes}
\graphicspath{{../Images/}}

\fancyhead[l]{Jonathan Cheng}
\fancyhead[c]{MCB 55}
\fancyhead[r]{September 25, 2024}
\fancyfoot[c]{Page \thepage\ of \pageref{LastPage}}

\begin{document}

\section*{Helminths - Parasitic Worms}
Approximately 1.5 billion people are infected yearly with soil-transmitted helminths worldwide

\tab \indicates Endemic worm infections (Many communities have 70-90\% of population CURRENTLY infected)

Most worms do not cause mortality but can result in morbidity (disease effects)

\tab \indicates Often asymptomatic and difficult to diagnose

\subsection*{Human-Human Transmission vs Zoonoses}
Human-Human:
\begin{enumerate}
    \item Humans can be definitive or intermediate hosts
    \item Some of these infections can cyclically go between humans and animals
\end{enumerate}

Zoonoses:
\begin{enumerate}
    \item Distinct animal reservoirs
    \item Humans usually infected by eating undercooked food, drinking ``contaminated'' water, and close interaction with domestic animals
\end{enumerate}

\subsection*{Location}
Size:
\begin{enumerate}
    \item Large enough to be seen with naked eye
    \item Makes it difficult to eliminate with normal immune function
    \subitem Often `ignored' by immune system rather than causing disease
\end{enumerate}

Most live in intestines

Outside of Intestines:
\begin{enumerate}
    \item Schistosoma worms are blood flukes (Adults reside in blood vessels)
    \item Guinea worms can live in peripheral tissues
    \item Intestinal worms can travel through lungs or other tissue as part of life cycle
    \item Can be found in an intracellular cyst form (particularly muscle or nervous tissue)
    \subitem Enhances transmission
\end{enumerate}

\subsection*{Nematode - Pinworm Enterobiasis}
Pinworm infection is an urban pathogen caused by the worm \textit{Enterobium Vermicularis}

Disease called enterobiasis (Most common worm infection in the US)

No animal reservoir (Primarily a disease of small children in child-care centers)

Diagnosed by itching and by examining for eggs by tape test (Fecal/oral transmission between humans)

\subsection*{Nematode - Ascaris Lumbricoides}
More than 1 billion people are infected at any given time

Adult females are up to 30 cm (males are smaller)

Large round worms that live inthe small intestine of infected hosts

\subsubsection*{Life Cycle}
\begin{enumerate}
    \item Ingested egg hatches in small intestine
    \item Larvae penetrates the intestine, travels through lymph and blood to liver, heart, and lungs
    \subitem Cannot attatch to intestine so instead moves to lung and gets coughed up and returns to intestine when larger (9 days after initial infection)
    \item Mature over next 8-12 weeks into sexually mature adults
\end{enumerate}

\subsubsection*{Clinical Issues}
\begin{enumerate}
    \item 807 million - 1.2 billion people in the world infeected with Ascaris
    \item Highly endemic in many tropical and subtropical areas
    \item Primary preventive issue is sanitary disposal of feces
    \item Most people with Ascaris worm infections have a low worm burden and are usually asymptomatic
    \item Even with heavy worm burden, most people are rarely diagnosed (Stomach problems or pneumonitis are not specific to Ascaris)
    \subitem Definitive diagnosis requires finding eggs in stool or worms expelled
\end{enumerate}

Most life-threatening complications involve blockage of intestine or bile ducts with worms

\tab \indicates Severe complications are rare

Only in cases with a heavy worm burden would adult worms exit nose, mouth, or anus

\subsection*{Nematode - Hookworms}
Ancylostoma parasites are transmitted through contaminated feces or from contact with largal form in soil

`Hook' onto intestinal wall with `teeth'

Ingest RBC for nutrition

Life Cycle:
\begin{enumerate}
    \item Larval form in soil can enter through unbroken skin
    \item Travel to intestine
    \item Mature into adult
    \item Find mate and produce eggs to pass in feces
\end{enumerate}

Higher level of disease because of blood loss

\subsection*{Nematode - Filarial Worms}
Definitive human host with mosquito vector for transmission

Cause lymphatic filariasis (elephantiasis)

Microfilariae are secreted by adults into blood for uptake by mosquito

\tab \indicates Can build up into lymphatic vessels (edema) and block fluid uptake along with other infections

High on list of neglected tropical diseases

\tab \indicates Most of work done by philanthropic organizations with drugs and bed nets

\subsection*{Nematode - Guinea Worm}
Dracunculiasis is caused by ingestion of drinking water with infected small crustaceans (copepods)

Infects humans, dogs, baboons, and cats

Female guinea worm extends anterior end through skin and forms a painful lesion up to 1 year after infection

When exposed to water, the female releases larvae into water

Removal by worm twisting (few centimeters per day up to a meter long to prevent breaking off under skin)

Guinea worm is close to eradication (education has been key to limit transmission)

\subsection*{Cestodes - Parasitic Worms}
Flat, segmented, and intestinal dwelling tapeworms

No mouth, alimentary tracts, circulatory system, body cavity, nor major metabolic pathways

\tab \indicates Absorb nutrient molecules through body surface (Attatches in small intestine)

Can be 4-10 meters in length

\subsubsection*{Anatomy}
Scolex: anterior portion of adult worm (head) and is armed with large hooks alternating with small ones

Neck is followed by a chain of flat, ribbon-like segmets referred to as proglottids

Infection by ingestion (primarily cycles between animals and humans)

\subsubsection*{Taeniasis}
Occurs when raw or undercooked beef (\textit{Taenia saginata}) or pork (\textit{T. solium}) are eaten

Humans carry the adult forms meaning they are definitive hosts

\tab \indicates However humans can be infected with cysts of \textit{T. solium} and therefore can also be intermetiate hosts

2010 WHO estimate that the global burden is 514,000 infections but cysts can cause the serious disease of cysticercosis

\subsubsection*{Cysts (Cysterceri)}
Can be found in any organ ranging from 5mm to 20cm

Cysts in muscle or liver leads to aches or inflammation

Cysts in eye or brain (neurocysticercosis) can cause blindness or neurologic damage

\tab \indicates Neurocysticercosis is the greatest cause of acquired epilepsy worldwide

\subsection*{Trematodes - Schistosomes}

Trematodes are commonly known as flukeworms

Snails are intermediate hosts or animal vector

Schistosomiasis (Bilharzia or Snail fever) is a parasitic disease caused by helminths from the genus Schistosoma

Blood fluke (\textit{Schistosome mansoni}) live in mesenteric (intestinal) veins

\tab \indicates Live for usually 4 years but can live up to 20 years

Found in up to 200 million people worldwide and is endemic to 76 countries throughout the tropics

\subsubsection*{Life Cycle}
Eggs released in urine or stool of infected individuals into fresh water

Asexual proliferation in snail to swim and penetrate skin

\subsubsection*{Schistosoma Disease}
Adult worms do not cause disease

People get bloated abdomen from excess fluid released into peritoneum from inflammation triggered by egg induced granulomas in tissues surrounding adult worms

Hepatic vs intestinal granuloma:
\begin{enumerate}
    \item Intestinal granulomas have less eosinophils, T cells, and B cells than hepatic granulomas
    \item More macrophages for intestinal granulomas
    \item During later stages, eggs in the liver become trapped and eggs in the gut seek to be released
\end{enumerate}





\end{document}