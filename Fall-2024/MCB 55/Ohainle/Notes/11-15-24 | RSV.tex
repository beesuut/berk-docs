\documentclass{notes}
\graphicspath{{../Images/}}

\fancyhead[l]{Jonathan Cheng}
\fancyhead[c]{MCB 55}
\fancyhead[r]{November 15, 2024}
\fancyfoot[c]{Page \thepage\ of \pageref{LastPage}}

\begin{document}

\section{Respiratory Syncytial Virus}

Family: \textit{Pneumoviridae}

\tab `Pneumon': Greek for lung

\tab Single-stranded, negative-sense RNA genome

Genus: \textit{Orthopneumovirus}

\tab Example Species: Human Respiratory Syncytial Virus (RSV)

\subsection{Characteristics of Orthopneumoviruses}

Lipid membrane + negative-sense RNA genome

Genome: 15 kb in size, encoding 11 proteins (Fairly large for RNA viruses)

Hosts: Humans, cattle, rodents (RSV only in humans and chimpanzees)

Transmission: Respiratory

Surface glycoproteins [F(usion) and G(lycoprotein)] are targets of host antibody responses

\subsection{RSV Pathogenesis and Transmission}

RSV infects upper (and sometimes lower) respiratory tract

\tab Re-infection is common

Young children: Lower respiratory tract infection (pneumonia, respiratory failure)

\tab Everyone is infected by age 2

\subsubsection{Seasonality}

Due to both environmental factors and human behavior

\tab Drastically decreased infections in 2020-21

\subsubsection{Severity}

RSV infections are most severe in babies \(\sim\)1 year

\tab Most susceptible to severe RSV illness

Vertically transmitted antibodies wane over time

\tab `Immunopathogenesis' likely plays a role

Low income, babies 0-4, and elderly 65+ are the most at risk

\subsection{Source of RSV}

First isolated from chimpanzees in 1955

Closely-related viruses in cows and more distantly-related in bats

\subsection{Challenges for Vaccine Development}

\begin{enumerate}
    \item Early age of infection
    \item Evasion of innate immunity
    \item Natural immunity does not prevent reinfection
    \item Vaccine-enhanced illness occurred with original RSV vaccine
\end{enumerate}

\subsubsection{Case Study: RSV Vaccine Candidate}

1960s: Formalin-inactivated vaccine

\tab Vaccine enhanced disease with subsequent infection in clinical trials

\tab Non-neutralizing antibodies resulted in immunocomplexes forming `inflammatory deposits' in lung tissue

\subsection{Replication Cycle}

\image{RSV Replication}{RSV Replication}{RSV Replication}{300pt}

The viral glycoprotein (envelope) changes shape (conformation) quite dramatically during fusion with the cell membrane

Good anti-RSV antibodies bind to specific places on the RSV F glycoprotein

\tab The F protein exists in multiple shapes

\subsubsection{Highly Effective RSV Vaccine}

Some key antibody binding sites are not found in the postfusion F protein

Making antibodies to the postfusion protein is not helpful and can be harmful 

\tab New RSV vaccine only immunizes against prefusion F protein

\tab Mutation introduced which `locks' the protein in the prefusion conformation

Moderna mRNA RSV vaccine: 83.7\% effective at preventing lower-respiratory disease

\tab Vaccinating pregnant women protects babies through passive immunization (mother to baby)

\subsubsection{Therapeutics}

Monoclonal antibody therapy: Antibodies similar to those elicited by vaccine

\tab Given to very high-risk infants

\end{document}