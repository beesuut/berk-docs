\documentclass{notes}
\graphicspath{{../Images/}}

\fancyhead[l]{Jonathan Cheng}
\fancyhead[c]{MCB 55}
\fancyhead[r]{November 6, 2024}
\fancyfoot[c]{Page \thepage\ of \pageref{LastPage}}

\begin{document}

\section{Poliovirus}

Family: \textit{Picornaviridae}

\tab `Pico' - Something small

\tab Single-stranded, positive-sense RNA genome

Genome: \textit{Enterovirus}

\tab Transmits through the intestine

\tab \tab Fecal-oral (or respiratory) transmission

Examples: Poliovirus, Enterovirus D68, Rhinovirus

\subsection{Picornaviruses}

\(\sim\)8 kb genome

Viral genome is `infectious'

\tab RNA is both mRNA and viral genome

Protein shell instead of lipid envelope (membrane)

\tab Very stable

Host species: Humans and other mammals

3 poliovirus `serotypes' are all variants that infect people

Disease: Paralysis (non-polio and polio-type), `summer cold', meningitis, diarrhea

Picornaviruses replicate in close asociation with lipid membranes of host cells

\tab Viral RNA replication machinery associated with membranes

\subsection{Proteins}

Picornaviruses make many individual proteins by breaking up (cleaving) a large `polyprotein' with a virally-encoded protease

\image{Picornavirus}{Picornavirus}{Picornavirus replication}{300pt}

Protease makes a good target for antibodies

\subsection{Poliovirus (The Disease)}

Mostly sporadic infection until 1905 when it became an epidemic

Poliovirus is an \textit{enterovirus}

\tab Spreads via fecal-oral but can also spread to and infect motor neurons

\begin{enumerate}
    \item Most infections are mild or asymptomatic
    \item Central nervous system infection in 0.5-1\% of cases resulting in paralysis of limbs (Poliomyelitis)
    \subitem 30\% of cases are permanent
    \item 40\% of those who recover suffer `post-polio syndrome' 30-40 years later
    \item 5-10\% of those paralyzed die when breathing muscles become immobile
\end{enumerate}

\subsection{Poliovirus Replication}

\begin{enumerate}
    \item Ingested polio replicates in oropharyngeal and intestinal mucosa
    \item Excreded in feces over a period of several weeks after infection
    \item Reaches the blood through the lymph nodes
    \item In some cases, can enter the central nervous system
    \subitem Cause of paralytic polio
    \item through retrograde axonal transport
    \subitem Stage blocked by antibodies (maternal or vaccination)
\end{enumerate}

\subsubsection{Success as a Pathogen}

\begin{enumerate}
    \item Only found in humans but is able to infect virtually all people
    \item No treatment once infected
    \item HIGHLY contagious
    \subitem Very stable in the environment (protein shell) and secreted for weeks or longer
    \item High number of asymptomatic infections
\end{enumerate}

\section{Protection Against Polio}

1910-1950: Summer in N hemisphere was seen as the season for polio (Due to relative humidity)

\tab Only prevention is avoiding contact or vaccination

\tab \indicates Closures of pools, schools, and public places

\subsection{Vaccine Development}

\subsubsection{HeLa Cells}

Henrietta Lacks died in 1951 from an aggressive cervical cancer

\tab Her cells were incredibly robust (immortal)

\tab \indicates Played a key role in dewvelopment of polio vaccines and many other biomedical studies

\subsubsection{Inactivated Poliovirus Vaccine (IPV)}

`Salk' Vaccine available in 1955

Inactivated virus: Killed (no replication)

Made by formalin(formaldehyde)-inactivation of wild type virus

\subsubsection{Attenuated Oral Poliovirus Vaccine (OPV)}

`Sabin' Vaccine available in 1959

Attenuated Virus: Live virus (can replicate)

\tab Just 2 mutations in serotypes 2 and 3 are sufficient for reversion to a virulent virus

Created through `attenuation'

\tab Attenuation seeks to isolate a virus that induces immunity, but not disease

\tab Attenuating mutations reduce initial viremia (more type for immune system to respond)

\begin{enumerate}
    \item 9 mutations for type 1 poliovirus
    \item 3 for type 2
    \item 5 for type 3
\end{enumerate}

\subsubsection{Reversion of Virulence}

Polio virus acquires 2\% nucleotiee divergence in the 5 days that it takes the virus to go from the mouth to the gut in one individual

OPV can mutate when it replicates and revert to a viruelent form known as vaccine-derived poliovirus (VDPV)

\tab OPV is no longer used in countries that have eradicated polio

\tab Causes Vaccine-Associated Paralytic Poliomyelitis (VAPP)

\subsection{Comparing the Vaccines}

\textbf{`Salk' Inactivated Polio Vaccine}

Advantages:

\begin{enumerate}
    \item No viral spread from vaccine
    \item No risk of vaccine-related poliomyelitis
    \item Induces serum antibodies that protect against infection of the CNS
\end{enumerate}

Disadvantages:

\begin{enumerate}
    \item Does not protect against infection of the intestine
    \item Vaccinated people can still be infected (but won't get poliomyelitis)
    \item Does not stop spread
    \item Needs to be injected (trained personnel)
    \item Cost (5x that of OPV plus cost of needles and trained health care worker)
\end{enumerate}

\textbf{`Sabin' Oral Polio Vaccine}

Advantages:

\begin{enumerate}
    \item Easy to administer without training (oral liquid)
    \item Cheap: Sabin assigned his rights to the vaccine strains over to the WHO which greatly helped with low-cost availability
    \item Replication in intestine induces mucosal immunity and prevents new infections
    \item Virus is shed (`contact immunity')
\end{enumerate}

Disadvantages:

\begin{enumerate}
    \item Virus is shed: infection of immunocompromised hosts or naïve populations
    \item But OPV can replicate in a vaccinee which means the virus can mutate
    \item Reversion to wild-type in gut: non-attenuated strain can infect other people
\end{enumerate}

\subsection{Poliovirus Eradication}

Reasons for vaccine success:

\begin{enumerate}
    \item No animal reservoir
    \item Two effective vaccines
    \item Little antigenic variation
    \item Cheap and easy to deliver OPV
\end{enumerate}

Polio remains epidemic only in Pakistan and Afghanistan

VDPV in New York

\tab Travel can lead to transmission into countries which have `eradicated' polio

\tab IPV does not replicate \indicates virus can still replicate and spread

New attenuated poliovirus vaccine may not revert to virulence

\end{document}