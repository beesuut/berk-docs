\documentclass{notes}
\graphicspath{{../Images/}}

\fancyhead[l]{Jonathan Cheng}
\fancyhead[c]{MCB 55}
\fancyhead[r]{November 13, 2024}
\fancyfoot[c]{Page \thepage\ of \pageref{LastPage}}

\begin{document}

\section{Flaviviruses}

Family: \textit{Flaviviridae}

Single-stranded positive-sense RNA genome

\tab `Flavi' - Yellow

Genus: `Flavivirus'

Example Species: Zika, Dengue, Yellow Fever, West Nile

\tab Aedes Aegypti \& Aedes Albopictus

\subsection{Characteristics of Flaviviruses}

Linear, ssRNA positive-sense genome

\tab \(\sim\)10 kb in size

Host species: Humans, mammals, insects

Transmission: vector-borne (mosquitoes)

Diseases: Hemorrhagic fever, encephalitis (brain inflammation)

\tab Surface `dimers' cover nearly entire virus

\image{Flavivirus}{Flavivirus}{E protein: Target of antibody recognition}{300pt}

\subsubsection{Genome}

Long genomic polyprotein is cleaved by a viral protease (like polio)

Genomic RNA looks like host mRNA

\section{Dengue Virus}

\subsection{Infection}

Dengue infects both humans and mosquitoes

\begin{enumerate}
    \item DENV replicates in people (4-7 day incubation) and infects mosquitoes
    \item DENV replicates in the mosquito vector (8-10 day incubation period)
    \item Mowquitoes transmit DENV to new people
\end{enumerate}

Dengue viruses circulate in both human (urban) and primate (sylvatic) cycles

\tab Aedes (human mosquito) sometimes transmits to monkeys

\tab Haemagogus (primate mosquito) sometimes gets infected by humans

Virus is detected from research in mosquitos, not monkeys

\subsection{Emergence}

Dengue is a zoonotic spillover from infected primates

\image{Dengue1234}{Dengue1234}{Primate Dengue is more closely related to human Dengue than other 3 serotypes of human Dengue}{330pt}

\subsection{Disease Burden}

\(\sim\)390 million total dengue infections annually

Geographical range of mosquito vectors has increased due to global connectivity and climate change

3/4 of Dengue infections are asymptomatic

\tab Few get Dengue fever

\tab \tab Fewer get Dengue Hemorrhagic Fever

\subsection{Pathogenesis}

Dengue Disease:

\tab High fever, nausia, vomiting, rash, aches and pains (eyes, joints, bones)

Self-limiting (most people recover without assistance)

Severe Dengue:

\tab 1:20 people get shock, internal bleeding, and death

\section{Dengue Viruses: 4 Serotypes}

Antibody responses to one serotype of DENV \underline{cross-react} with the other serotypes

After first exposure, durable protection is provided to the particular serotype, with waning protection to other serotypes

\subsection{Antibody-Dependent Enhancement}

Occurs when antibodies generated during an immune response recognize and bind to a pathogen, but they are unable to prevent infection and instead enhance infection

\subsubsection{Mechanism}

FcGamma Receptors (FcgR) on many immune cells bind to antibodies

\tab DENV can enter (infect) immune cells using the antibodies

\tab DENV infection via this route may also result in aberrant cytokine secretion

\subsubsection{Vaccine Development}

Live attenuated DENV vaccine exists which includes all 4 DENV serotypes

Limitation: Vaccination acts as primary infection, leading to increased risk for ADE

\tab Was, however, extremely protective in people who had already been previously infected

Vaccine does not induce cytolytic T cell responses

\subsubsection{Mosquito Bacterium}

Wolbachia naturally infects many mosquito species but not Aedes Aegypti

\tab DENV transmission decreases in mosquitos with Wolbachia

\section{Zika Virus}

A DENV-like flavivirus that infects humans (significant immune cross-reactivity)

Typically mild (fever, rash, headaches, joint pain, etc.) or asymptomatic

Severe infections can occur in pregnant women

\image{Zika Transmission}{Zika}{Sexual and vertical transmission (can infect placenta)}{300pt}

Zika antibodies can increase risk of DENV disease (cross-reactivity)

Many places around the globe can circulate DENV and ZIKV

\tab However, very little Zika virus currently circulating globally

\end{document}