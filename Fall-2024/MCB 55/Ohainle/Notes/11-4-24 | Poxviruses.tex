\documentclass{notes}
\graphicspath{{../Images/}}

\fancyhead[l]{Jonathan Cheng}
\fancyhead[c]{MCB 55}
\fancyhead[r]{November 4, 2024}
\fancyfoot[c]{Page \thepage\ of \pageref{LastPage}}

\begin{document}

\section{Types of Viral Genomes (Baltimore Classification)}

All types of viruses make mRNA (messenger RNA) which allows them to make proteins

\(+\) sense: The genome can be translated directly into proteins

\(-\) sense: Other strand must be transcribed (turned into + sense) to make viral proteins

The type of genome can give clues as to how the virus replicates and evolves

\image{Genomes}{Genomes}{Viral genomes}{360pt}

\section{Poxviruses}

Family: \textit{Poxviridae}

Double Stranded DNA genome

\tab Replicates in the cytoplasm of the cells

Extremely large virus (Virion can be seen with just light microscopy)

Subfamily: \textit{Chordopoxvirinae}

Genus: \textit{Orthopoxvirus}

Examples: Vaccinia virus, Cowpox virus, Mpox (Monkeypox) virus, Raccoonpox virus, Variola virus

Note: First vaccine and the only human virus that has ever been completely eradicated

\subsection{Genome}

Many RNA viruses are \(\sim\)10 kb in size

Poxviruses have 170-250 kb of a linear, dsDNA genome

\begin{enumerate}
    \item Tightly packed genome
    \item Genes are not overlapping and do not have introns
    \item Replicate in the cell cytoplasm (no splicing)
    \item Encode many immunomodulatory genes
    \item Over 100 ORFs (open reading frames) (genes that encode proteins)
\end{enumerate}

\subsubsection{Immunomodulation}

Infected cells that sense viral infection release interferons (IFNs)

Poxviruses make a decoy receptor, binding to the IFN (similar to a neutralizing antibody)

\section{Smallpox}

Transmission: Respiratory droplets, `fomites', close contact

Lethality: \(\sim\)30\% plus significant scarring

Ancient disease of humans

\begin{enumerate}
    \item Over 3000 years ago in India, China, and Africa
    \item Endemic in Asia by 1000 A.D.
    \item Endemic in every European country by the end of the 19th century
    \item Large epidemics in Europe (1200-1600) and the Americas (1507-1524)
\end{enumerate}

Ex: Estimated 40\% of Tenochtitlan (pop: 200,000) died from Smallpox in 1520

Ancient remains can be used to sequence and date Smallpox using phyogenetics

\tab Viking age Variola Virus discovered in 2020

\subsection{Variolation}

Early form of vaccination for Smallpox

16th century: First reference to inoculation with dried pox in China (likely also used in India)

Variolation: Dried material from infected persons ground up and placed under skin or inhaled

\tab Risk of Death: 14\% \indicates 2\%

Risks: 2-3\% of people develop severe smallpox infection and spread disease

\subsection{Early Vaccines}

1790s: Observation that milkmaids exposed to cowpox did not develop smallpox

Edward Jenner inoculated his gardener's son with cowpox and showed that the boy did not become sick after infection with smallpox scabs

\tab Called the process vaccination (vacca: latin for cow)

\subsection{Vaccinia Virus}

Poxviruses infect a wide range of mammalian species

Smallpox Vaccine: Live-attenuated viruses with unknown origins and attenuation history

\tab Vaccinia virus-MVA is more closely related to monkeypox than smallpox

\subsection{Orthopoxvirus Diversity}

Cross-species transmission of poxviruses can occur

\begin{enumerate}
    \item Cowpox from cows to humans
    \item Monkeypox from small mammals to humans
    \item Buffalopox from water buffaloes to humans
    \item Spread within humans limited after transmissions
\end{enumerate}

Despite genetic diversity, infection with one orthopoxvirus can confer protective immunity against another

\subsection{Eradication}

Smallpox officially declared eradicated in 1980

\tab Led to the end of smallpox vaccination

\tab \tab \(>\)70\% of the global population is not immune to smallpox (or other related poxviruses)

\subsection{Mpox}

Can trigger fevers, aches, and painful fluid-filled skin lesions (can be fatal)

Mpox endemic in parts of Africa (Likely a rodent reservoir / spillover in 2022)

Endemic: Majority non-sexual transmission

2022 (Americas and Europe): Majority sexual encounters in gay populations

\subsubsection{Case Fatality Rate (CFR)}

\image{CFR}{CFR}{Different versions of MPXV}{230pt}

Notes:

\begin{enumerate}
    \item Not all infections are detected
    \item Less severe cases not documented
    \item Other heath conditions
\end{enumerate}

\subsubsection{Modern Vaccines}

Smallpox vaccine protects against Mpox infection (Relatedness and immune cross-reactivity)

Current Challenge: Lack of vaccine supply where it is needed (Ex: DRC)


\end{document}