\documentclass{notes}
\graphicspath{{../Images/}}

\fancyhead[l]{Jonathan Cheng}
\fancyhead[c]{MCB 55}
\fancyhead[r]{November 8, 2024}
\fancyfoot[c]{Page \thepage\ of \pageref{LastPage}}

\begin{document}

\section{Measles}

Family: \textit{Paramyxoviridae}

`Myxa' - Greek for mucus

Single-stranded, negative-sense RNA genome

Genus: \textit{Morbillivirus}

Example Species: Measles, Rinderpest, Ceacean morbillivirus

`Morbilli' - Latin for `little disease'

\tab `Big disease' was the plague

Measles - Diminutive of Latin `misella' (misery)

\tab Ancient infection, most closely related to rinderpest (Cow morbillivirus)

\tab \tab Possibly a spillover during the domestication of cattle

\subsection{Morbillivirus Characteristics}

Single-stranded RNA genome

\(\sim\)15-16 kb

Host species: Humans, dogs, cattle, cetaceans (whales, dolphins, etc.)

Transmission: Respiratory

Diseases: Measles (fever, rash, cough, diarrhea), SSPE (chronic brain infection, immunosuppression)

\image{Measles}{Measles}{Measles (lipid envelope in pink)}{270pt}

\subsection{Measles Virus Spread}

\(\sim\) 2 week incubation period

\begin{enumerate}
    \item MV enters the airway and infects macrophages and dendritic cells
    \item Infected cells move the virus to lymph nodes
    \subitem Virus spreads to additional lymph tissues and organs
    \item Infection spreads to the epithelium in the airway
    \item Progeny viral particles are released in the trachea and expulsed by coughing and sneezing
\end{enumerate}

\subsection{Pathogenesis}

\begin{enumerate}
    \item Infection and severe depletion of activated and memory T and B cells
    \subitem \underbar{Immunosuppression} thought to be a direct result of this lymphoid cell killing
    \subitem `Immune amnesia'
    \item Complications or death by opportunistic infections
    \item Severe immunosuppression can last several years
\end{enumerate}

\subsubsection{Immune Amnesia}

Measles virus infection causes elimination of the antibody repertoire globally

This massively wipes out preexisting immunity against other pathogens

\tab \indicates Wipes out vaccine-generated immunity

\subsection{Subacute Sclerosing Panencephalitis (SSPE)}

Rare chronic infection in brain (fatal)

\tab May be due to the stability of RNA / nucleocapsid complexes inside cells

Associated with cognitive decline, impaired motor functions, seizures

\subsection{\texorpdfstring{R\(_0\): Basic Reproduction Number}{R0: Basic Reproduction Number}}

A figure expressing the average number of cases of an infectious disease will arise by transmission from a single individual (in a population that has not previously encountered the disease)

R\(_0 < 1\): Disease will decline

R\(_0 > 1\): Disease will spread (outbreak, epidemic, or pandemic)

Measles is one of the most contagious viruses today

\image{R0}{R0}{Measles has R\textsubscript{0} of 15}{250pt}

\subsection{Immunity}

Measles immune memory is strong and long-lasting

Ex: Measles epidemic in Faroe islands (1791)

\tab No measles infection for the next 65 years

\tab When measles returned, \(>\)75\% of the population was infected, but most elderly residents were not

Measles vaccine is now given at 9 months and 5 years as a part of a trivalent MMR (measles, mumps, rubella) vaccine

\subsubsection{Herd Immunity}

In a non-imune (naive) population, a pathogen will spread quickly

Population-level immunity builds over time as infections and vaccines spread

\tab Herd immunity can be achieved through infection or vaccination

Measles has a 91-94\% `herd immunity threshold,' making eradication difficult

\tab Smallpox required 80-85\%

\subsubsection{Immunizations}

Disneyland outbreak had a large positive impact on vaccination rates

\tab Conditional admission for kindergarteners who were not up to date on vaccinations

Many global measles immunization campaigns have been cut off due to SARS-CoV-2 pandemic

New delivery modality: Plastic disc with `microneedles' delivers vaccination easily

\end{document}