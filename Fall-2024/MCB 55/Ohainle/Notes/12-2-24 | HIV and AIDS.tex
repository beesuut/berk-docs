\documentclass{notes}
\graphicspath{{../Images/}}

\fancyhead[l]{Jonathan Cheng}
\fancyhead[c]{MCB 55}
\fancyhead[r]{December 2, 2024}
\fancyfoot[c]{Page \thepage\ of \pageref{LastPage}}

\begin{document}

\section{Timeline of HIV in the US}

\begin{enumerate}
    \item \textbf{June 1981:} MMWR report
    \item \textbf{September 1982:} AIDS defined
    \subitem ``a disease, at least moderately predictive of a defect in cell-mediated immunity, occurring in a person with no known cause for diminished resistance to that disease''
    \item \textbf{December 1982:} Case of AIDS after blood transfusion reported
    \subitem 1985: FDA instituted lifetime ban on blood donation for MSM
    \item \textbf{March 1983:} CDC addresses transmission risks
    \subitem \textit{`AIDS may be caused by an infectious agent that is transmitted sexually or through exposure to blood products'}
    \item \textbf{January 1983:} `Ward86' opens at UCSF as the first outpatient AIDS clinic
    \item \textbf{June 1983:} AIDS activism begins
    \item \textbf{May 1983:} HIV is discovered as the virus which causes AIDS (Nobel Prize)
    \subitem \textbf{January 1985:} Conclusively determined that AIDS caused by HIV
    \subitem 1985 First public mention of AIDS by Reagan
    \item \textbf{March 1987:} ACT UP (AIDS Coalition to Unleash Power) forms grassroots political group to end AIDS pandemic
    \item \textbf{March 1987:} FDA approves first antiretroviral drug (Zidovudine = `AZT')
    \item \textbf{1992:} AIDS becomes \#1 cause of death for US men aged 25-44
\end{enumerate}

\section{HIV Viral Lifecycle}

\subsection{Reverse Transcription and Integration}

\begin{enumerate}
    \item gp120 on the virus attaches to CD4 receptor on T cells
    \item gp41 mediates membrane fusion: viral capsid and RNA genome are deliviered
    \item RNA genome is \textbf{reverse transcribed} into DNA and the DNA is then \textbf{integrated} into the T cell genome by integrase
    \item The `Provirus' is the integrated DNA copy of the RNA genome
    \subitem Integrated virus can remain silent for a long time (\textbf{latent})
\end{enumerate}

Retroviruses stably integrate DNA into host cell chromosomes

\begin{enumerate}
    \item Persistence: Infection is not easily cleared
    \item Latency: Integrated, but silent proviruses are not seen by the immune system
    \item Normal cell division increases number of infected cells (clonal expansion)
\end{enumerate}

\tab \indicates Reason why HIV is chronic and has no cure

\subsection{Mutation Rate}

Our DNA polymerases that replicate our DNA: 1 in 1 billion base pairs mutate per cell division (3 mutations per cell division)

HIV reverse transcriptase: 1 in 10,000 base pairs mutate per replication (HIV RNA genome is around 10,000 base pairs)

\tab Approximately one mutation every time HIV replicates

\tab Low estimate of release into blood of virions is 10 billion per day

\tab On average every possible mutation and every possible position in the HIV genome is predicted to occur each day

\tab Since the start of the HIV-1 group M pandemic (early 1900s), HIV sequences have massively diverged

\image{HIV Mutation}{HIV Mutation}{HIV mutation in a single individual is equivalent to that of an entire flu season}{300pt}

\subsection{Assembly and Release}

\begin{enumerate}
    \item The integrated provirus directs the creation of new RNA viral genomes (by transcription) and new viral proteins (by translation)
    \item HIV Protease cuts up a precurser protein (e.g. Pol) into smaller functional proteins (reverse transcriptase, protease, integrase)
    \item New viral particles are then assembled and released from the cell to infect new cells (`budding')
\end{enumerate}

\image{Protease}{Protease}{Protease in the HIV life Cycle (Maturation)}{230pt}

\image{Maturation}{Maturation}{Mature and infectious virions}{230pt}

\section{HIV Pathogenesis}

HIV infection has 2 stages: Acute and Chronic

Acute infection:

\begin{enumerate}
    \item flu-like symptoms (50-90\% of people)
    \item 2-4 weeks after infection, lasting a few days or weeks
\end{enumerate}

Chronic infection:

\begin{enumerate}
    \item Severe immunodeficiency \indicates AIDS
    ]item CD4 T cells are lost by direct killing by virus and immune mechanisms
\end{enumerate}

AIDS is a disease of helper CD4 T cells

\tab Allows recruitment and activation of immune cells, helps CD8 T cells, and helps B cells make antibodies

The immune system is damaged during all infections, but in untreated people, the virus is chronic adn the disease is progressive

Destruction of the immune system leads to \underline{opportunistic infections}

\subsection{Acute Phase}

Initial peak of viremia (virus in plasma) with flu-like symptoms

\subsection{Chronic Phase}

Viremia is controlled and reaches a low level

Number of CD4+ T cells slowly declines as virus / immune responses kill them

\tab CD4+ T cells reach low enough level that the immune system begins to fail

\tab Virus replicates out of control \indicates Further loss of immune function \indicates opportunistic infection

\image{HIV Pathogenesis}{HIV Pathogenesis}{HIV Pathogenesis}{200pt}

AIDS is the result of depleted CD4+ T cells and a severely compromised adaptive immune system

\indicates Severe form of \underline{immunosuppression}

\tab \indicates TB is the leading cause of death globally among people living with HIV (opportunistic infection in PWH [People with HIV])

\section{HIV Treatments}

\subsection{Antiretroviral Therapy (ART)}

1987 AZT is the first drug to treat HIV infection (RT inhibitor)

\tab Side effects and resistance development were major issues

Antiretroviral therapy targets multiple steps of the HIV lifecycle

\begin{enumerate}
    \item Binding and fusion
    \item Uncoating (New drug)
    \item Reverse Transcriptase (AZT)
    \item Integrase
    \item Protease (No Maturation)
\end{enumerate}

When used in combination, virus cannot as easily evolve to escape (but cannot fully cure HIV)

\tab Known as Highly-Active Antiretroviral Therapy (HAART)

\image{HIV Treatment}{HIV Treatment}{Treatment as prevention is more than 96\% effective}{300pt}

Although the virus can still take over if therapies are stopped, suppressed virus cannot transmit

ART blocks viral replication, but is not a cure (can't remove integrated provirus)

\tab Can prolong the life expectancy of PWH to near normal and can prevent transmission

\tab \indicates Still have a higher risk for certain cancers / neurocognitive disorders

\subsection{Current Status}

Important that people know their HIV-positive \textbf{status}, are receiving \textbf{treatment}, and are virally \textbf{suppressed}

\subsection{Pre-Exposure Prophylaxis (PrEP)}

Can reduce chance of getting HIV from sex (99\%) or injection drug use (\textgreater74\%)

\tab Requires high levels of adherence and doesn't protect against other STDs

Can be (daily) pills or shots (every 2 months)

\begin{enumerate}
    \item Descovy or Truvada: Inhibits reverse transcriptase
    \item Apretude: Integrase inhibitor
    \item Lenacapivir (new): targets HIV capsid
\end{enumerate}

Global access to most effective PreP (Lenacapavir) is a major challenge

\subsection{Treatment as Prevention}

U = U (Undetectable = Untransmissible)

More widespread use of PrEP

UNAIDS Goals: 90:90:90 by 2025 and 95:95:95 by 2030

Better antiretroviral therapies:

\begin{enumerate}
    \item Longer acting drugs
    \item Lower toxicity
    \item Better resistant profiles
    \item More convenient combinations
    \item Broadly neutralizing antibodies under development
\end{enumerate}

\subsection{HIV Cure?}

Several people have been `cured' of HIV

Why a vaccine?

\begin{enumerate}
    \item Long-lasting protection
    \item Cost-effective
    \item Community immunity
    \item Adherence / stigma not an issue
\end{enumerate}

Why is there no HIV vaccine?

\begin{enumerate}
    \item HIV may be hard to generate immunity to (few envelope proteins)
    \item Don't know what kind of immunity should be elicited
    \item HIV mutability and variability
\end{enumerate}

(Very) few cases of sustained remission off ART suggest that a cure may be possible

\tab Due to stem cell transplant (replaces immune system with that of donor)

Some people have a mutation in CCR5 gene (surface receptor) meaning HIV cannot enter cells

\tab Hope for CRISPR gene editing?

\end{document}