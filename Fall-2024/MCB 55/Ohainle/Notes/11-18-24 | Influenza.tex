\documentclass{notes}
\graphicspath{{../Images/}}

\fancyhead[l]{Jonathan Cheng}
\fancyhead[c]{MCB 55}
\fancyhead[r]{November 18, 2024}
\fancyfoot[c]{Page \thepage\ of \pageref{LastPage}}

\begin{document}

\section{Influenza Viruses}

Family: \textit{Orthomyxoviridae}

Single-stranded, negative-sense RNA genome

Genus: \textit{Alphainfluenzavirus}, \textit{Betainfluenzavirus}, \textit{Gammainfluenzavirus}, \textit{Deltainfluenzavirus}

\subsection{Shapes}

Influenza viruses form either enveloped, spherical, or filamentous particles

\subsection{Life Cycle}

\begin{enumerate}
    \item Entry: HA glycoprotein
    \item Gene segments imported to nucleus
    \item mRNA are made
    \item mRNA is exported from the nucleus and is translated
    \item genome segments are packaged and virus buds from the cell
\end{enumerate}

\subsection{Characteristics of Orthomyxoviruses}

\begin{enumerate}
    \item Negative-sense RNA genome
    \item Segmented genome
    \item Genome: 13.5 kb in size
    \subitem Encodes 11 proteins
    \item HA (Hemagglutinin) and NA (Neuraminidase) surface proteins: targets of antibodies
    \item Hosts: Aquatic birds, humans, pigs, horses, seals, cows, etc.
    \item Transmission: Respiratory (mammals), Fecal-oral (birds)
\end{enumerate}

\image{Influenza}{Influenza}{Orthomyxoviruses}{240pt}

Fish may have been ancient hosts

\tab \indicates Fish influenza viruses are closely related to all influenzas that infect mammals

\subsection{Types of Influenza}

4 general types of Influenza Viruses

\begin{enumerate}
    \item A and B viruses cause the flu season
    \subitem A viruses are the only influenza viruses known to cause flu pandemics
    \item C viruses cause mild illness and are not thought to cause human flu epidemics
    \item D viruses primarily affect cattle and are not known to infect or cause illness in people
\end{enumerate}

\subsection{Influenza A}

Most important influenza virus for human health

\tab Can cause local epidemics or pandemics with significant infection rates

\tab Wide host range and epidemiology involves close contact of humans, farm animals, and birds

Zoonotic spillover plays an important role of influenza virus biology

\subsection{Neuraminidase (NA)}

Needed for release of virions from the infected cell

\tab Virions have NA on their surface which allows for cleavage of sialic acid on the cell surface

\tab Target for the major antiviral agents against influenza

\subsubsection{Sialic Acid}

Sugars attatched to proteins (glycoproteins) and lipids (glycolipids)

Glycocalyx: dense, gel-like meshwork surrounding the cell (physical barrier)

Sialic acid: sugar component of the glycocalyx that influenza uses to bind to and enter cells

\subsection{Infection}

Influenza infects the respiratory epithelium in the lungs

\tab Upper respiratory tract infection: less severe

\tab Lower respiratory tract infection: more severe

Virus replication peaks \(\sim\)48 hours after infection and declines slowly thereafter

\tab \indicates Little shedding of virus after days 6 to 8

\image{Cytokine Storm}{Cytokine Storm}{Cytokine Storm: Immune system causes more pathology than virus itself}{240pt}

\subsection{Mortality}

Influenza has over 1 billion cases of seasonal flu annually

\tab 3-5 million severe cases

\tab Half a million deaths globally

\tab \tab 10s of thousands die each year from influenza in the US

Influenza mortality in the US has declined

\tab Likely due to behavior, hygiene, and vaccination

\subsection{Seasonality}

More Flu A than Flu B

Lowest rates of infection in summer, rises through fall and winter (like RSV)

\tab Dependent on temperature (northern and summer hemisphere are offset)

\tab Influenza is less seasonal near the equator

\tab \indicates Viruses at the Equator seed seasonal epidemics in N and S hemisphere

\subsection{Reassortment}

Influenza genes are broken into 8 segments

If 2 different influenza viruses infect the same cell, new viruses can be made of gene segments from both viruses

\tab \indicates Segmentation of the genome allows for \underline{reassortment}

\subsubsection{Reassorted Viruses Can Cause Major Pandemics}

\tab Ex: The 1918 flu looked different than influenza A viruses that were circulating in the population before

\image{Historic Influenza}{Historic Influenza}{Historic Influenza}{250pt}

\end{document}