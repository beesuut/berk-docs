\documentclass{notes}
\graphicspath{{../Images/}}

\fancyhead[l]{Jonathan Cheng}
\fancyhead[c]{MCB 55}
\fancyhead[r]{November 18, 2024}
\fancyfoot[c]{Page \thepage\ of \pageref{LastPage}}

\begin{document}

\section{Influenza Viruses}

Family: \textit{Orthomyxoviridae}

Single-stranded, negative-sense RNA genome

Genus: \textit{Alphainfluenzavirus}, \textit{Betainfluenzavirus}, \textit{Gammainfluenzavirus}, \textit{Deltainfluenzavirus}

\subsection{Shapes}

Influenza viruses form either enveloped, spherical, or filamentous particles

\subsection{Life Cycle}

\begin{enumerate}
    \item Entry: HA glycoprotein
    \item Gene segments imported to nucleus
    \item mRNA are made
    \item mRNA is exported from the nucleus and is translated
    \item genome segments are packaged and virus buds from the cell
\end{enumerate}

\subsection{Characteristics of Orthomyxoviruses}

\begin{enumerate}
    \item Negative-sense RNA genome
    \item Segmented genome
    \item Genome: 13.5 kb in size
    \subitem Encodes 11 proteins
    \item HA (Hemagglutinin) and NA (Neuraminidase) surface proteins: targets of antibodies
    \item Hosts: Aquatic birds, humans, pigs, horses, seals, cows, etc.
    \item Transmission: Respiratory (mammals), Fecal-oral (birds)
\end{enumerate}

\image{Influenza}{Influenza}{Orthomyxoviruses}{240pt}

Fish may have been ancient hosts

\tab \indicates Fish influenza viruses are closely related to all influenzas that infect mammals

\subsection{Types of Influenza}

4 general types of Influenza Viruses

\begin{enumerate}
    \item A and B viruses cause the flu season
    \subitem A viruses are the only influenza viruses known to cause flu pandemics
    \item C viruses cause mild illness and are not thought to cause human flu epidemics
    \item D viruses primarily affect cattle and are not known to infect or cause illness in people
\end{enumerate}

\subsection{Influenza A}

Most important influenza virus for human health

\tab Can cause local epidemics or pandemics with significant infection rates

\tab Wide host range and epidemiology involves close contact of humans, farm animals, and birds

Zoonotic spillover plays an important role of influenza virus biology

\subsection{Neuraminidase (NA)}

Needed for release of virions from the infected cell

\tab Virions have NA on their surface which allows for cleavage of sialic acid on the cell surface

\tab Target for the major antiviral agents against influenza

\subsubsection{Sialic Acid}

Sugars attatched to proteins (glycoproteins) and lipids (glycolipids)

Glycocalyx: dense, gel-like meshwork surrounding the cell (physical barrier)

Sialic acid: sugar component of the glycocalyx that influenza uses to bind to and enter cells

\subsection{Infection}

Influenza infects the respiratory epithelium in the lungs

\tab Upper respiratory tract infection: less severe

\tab Lower respiratory tract infection: more severe

Virus replication peaks \(\sim\)48 hours after infection and declines slowly thereafter

\tab \indicates Little shedding of virus after days 6 to 8

\image{Cytokine Storm}{Cytokine Storm}{Cytokine Storm: Immune system causes more pathology than virus itself}{240pt}

\subsection{Mortality}

Influenza has over 1 billion cases of seasonal flu annually

\tab 3-5 million severe cases

\tab Half a million deaths globally

\tab \tab 10s of thousands die each year from influenza in the US

Influenza mortality in the US has declined

\tab Likely due to behavior, hygiene, and vaccination

\subsection{Seasonality}

More Flu A than Flu B

Lowest rates of infection in summer, rises through fall and winter (like RSV)

\tab Dependent on temperature (northern and summer hemisphere are offset)

\tab Influenza is less seasonal near the equator

\tab \indicates Viruses at the Equator seed seasonal epidemics in N and S hemisphere

\subsection{Reassortment}

Influenza genes are broken into 8 segments

If 2 different influenza viruses infect the same cell, new viruses can be made of gene segments from both viruses

\tab \indicates Segmentation of the genome allows for \underline{reassortment}

\subsubsection{Reassorted Viruses Can Cause Major Pandemics}

\tab Ex: The 1918 flu looked different than influenza A viruses that were circulating in the population before

\image{Historic Influenza}{Historic Influenza}{Historic Influenza}{250pt}

\section{Zoonoses}

The reservoir of influenza A diversity resides in aquatic birds

\tab All subtypes of HA and NA are found in waterfowl / bats

\tab \indicates 16 subtypes of HA; 9 subtypes of NA

Only H1, H2, H3, N1, N2 have become endemic in humans

\subsection{Transmission Differences}

Influenza grows in the intestinal tract of aquatic birds and causes little to no disease

Influenza binds to different types of sialic acid, determining what cells can be infected

\tab Types of sialic acid are determined by mutations in HA (receptor binding)

\subsection{Avian Influenza}

Avian HA uses a slightly different sialic acid compared to human-infecting influenza HAs

\tab Sialic acid used by birds is limited to the lower airways in huamns

\tab \indicates Leads to lower transmission efficiency but higher virulence

`Avian Flu' infections typically see higher viral loads and increased levels of chemokines / cytokines (cytokine storm) produced by bronchial epithelial cells and alveolar macrophages

\image{Avian Flu}{Avian Flu}{Avian influenza segments transferred to humans}{250pt}

\subsection{2009 H1N1 `Swine Flu' Pandemic}

Unusual cluster of illnesses very late in the flu season (March-April)

Spread world-wide by fall of 2009

\tab Vaccine came late but severity was not super high

Came about as a hybrid from human, avian, and swine viruses

Unusual due to lack of morbidity in persons \(>\)65 years old

\tab Cross-reactive immunity: Some antigenic similarity with much older strains

\tab \(\sim\frac{1}{3}\) had neutralizing antibodies

2009 H1N1 still circulates as a seasonal influenza

\section{Where Might The Next Pandemic Influenza Come From?}

\image{Flu Reservoirs}{Flu Reservoirs}{Influenza Reservoirs}{250pt}

New influenza serotype (H7N9) in China (2013-2017) due to direct contact with poultry

\begin{enumerate}
    \item Closing of poultry market
    \item High morbidity and mortality
    \item Some human-human transmission but primarily direct contact with poultry
    \item Complex reassortment with segments from wild birds, domestic waterfowl, and chickens
    \item Controlled by culling birds and immunizing chickens
    \item May have been a pandemic averted
\end{enumerate}

H5N1 Avian influenza (Avian segments only, without reassortment)

\begin{enumerate}
    \item VERY virulent in domestic poultry
    \subitem Highly-Pathogenic Avian Influenza (HPAI)
    \item Dramatic and unpredictable spread among wild and domestic birds
    \item Can, but does not always, have exceptionally high mortality rate
    \item None / limited human-to-human spread (so far)
\end{enumerate}

Significant increases in H5 infections since 2019 in poultry and wild birds

\tab H5N1 is a panzootic virus: Widely distributed infectious disease of animals

HPAI has recently become widespread in 25 wild mammalian species

\tab Most recently dairy cows (with mild symptoms)

\tab Reassortment or mutations are cause for worry

\tab \indicates Seasonal flu vaccines are given to workers in close contact with dairy cows

\end{document}