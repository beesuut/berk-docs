\documentclass{notes}
\graphicspath{{../Images/}}

\fancyhead[l]{Jonathan Cheng}
\fancyhead[c]{The Everything Document}
\fancyhead[r]{October 8, 2024}
\fancyfoot[c]{Page \thepage\ of \pageref{LastPage}}

\begin{document}

\section{Vectors}

\subsection{Dot Product}

The dot product is defined as

\begin{equation}
    \vec{A} \cdot \vec{B} = AB \cos(\theta) = A_1B_1 + A_2B_2 + A_3B_3 + \cdots
\end{equation}

where \(\theta\) is the angle between \(\vec{A}\) and \(\vec{B}\) when drawn tail to tail

\begin{align*}
    \vec{A} \cdot \vec{B} =& A \text{ times the projections of } \vec{B} \text{ onto } \vec{A} \\
    =& A \text{ times the projections of } \vec{A} \text{ onto } \vec{B}
\end{align*}

\subsection{Cross Product (Vector Product)}

The cross product is defined as

\begin{equation}
    \vec{C} = \vec{A} \times \vec{B} = AB \sin(\theta)
\end{equation}

and is non-commutative.

\subsubsection{Right Hand Rule}

The resultant vector \(\vec{C}\) from crossing \(\vec{A}\) (thumb) with \(\vec{B}\) (index) is the direction of the palm (middle)

\section{Kinematics}

\subsection{Velocity and Acceleration}

\begin{align}
    \vec{v} =& \deriv{r}{t} = \deriv{(x\uvec{i}+y\uvec{j}+z\uvec{k})}{t}\\[5pt]
    \vec{a} =& \deriv{\vec{v}}{t} = \deriv{v_x}{t}\uvec{i} + \deriv{v_y}{t}\uvec{j} + \deriv{v_z}{t}\uvec{k} \\[6pt]
     =& \deriv{^2\vec{r}}{t^2}
\end{align}

\subsection{Kinematics Equations}

\begin{gather}
    \vec{r} = \vec{r_0} + \vec{v_0}t + \frac{1}{2} \vec{a}t^2\\
    \vec{r} = \vec{r_0} + \frac{1}{2} (\vec{v_0}+ \vec{v})t\\
    \vec{v} = \vec{v_0} + \vec{a}t\\
    \vec{v}^{\,2} = \vec{v_0}^{\,2} + 2\vec{a}\cdot \Delta\vec{x}
\end{gather}

\section{Polar Coordinates}

\begin{gather}
    r = \sqrt{x^2+y^2} \\
    \theta = \arctan(\frac{y}{x})\\
    x = r\cos(\theta)\\
    y - r\sin(\theta)
\end{gather}

Defining the unit vectors,
\begin{gather}
    \uvec{r}(\theta) = \cos(\theta)\uvec{i} + \sin(\theta)\uvec{j}\\
    \uvec{theta}(\theta) = -\sin(\theta)\uvec{i} + \cos(\theta)\uvec{j}
\end{gather}

\begin{align}
    \deriv{\uvec{r}}{t} =& \, \dot{\theta} \uvec{theta}\\[4pt]
    \deriv{\uvec{theta}}{t} =& -\dot{\theta} \uvec{r}
\end{align}

Giving velocity and acceleration in polar coordinates:

\begin{gather}
    \vec{v} = \dot{r}\uvec{r} + r \, \dot{\theta} \uvec{theta}\\
    \vec{a} = (\ddot{r}-r(\dot{\theta})^2)\uvec{r} + (r\ddot{\theta} + 2\dot{r}\dot{\theta})\uvec{theta}
\end{gather}

\subsection{Unform Circular Motion}

\begin{gather}
    \dot{\vec{r}} = 0\\
    \ddot{\vec{\theta}} = 0\\
    \vec{v} = \omega r \, \uvec{theta}\\
    \vec{a} = -\omega r^2 \uvec{r} = -\frac{v^2}{r} \uvec{r}
\end{gather}

\section{Taylor Series}

General Form:

\begin{align}
    f(x) =& f(0)+ f'(0)x+f''(0)\frac{x^2}{2!} + \cdots\\
    f(a+x) =& f(a) + f'(a)x + f'(a)\frac{x^2}{2!} + \cdots\\
    \sin(x) =& x-\frac{1}{3!}x^3 + \frac{1}{5!}x^5 + \cdots = x\\
    \cos(x) =& 1-\frac{1}{2!}x^2 + \frac{1}{4!}x^4 + \cdots = 1-\frac{1}{2}x^2\\
    e^x =& 1+x+ \frac{1}{2}x^2+\frac{1}{3!}x^3 + \cdots\\
    \frac{1}{1\pm x} =& 1 \mp x + x^2 \mp x^3 + \cdots\\
    \frac{1}{\sqrt{1+x}} =& 1 - \frac{1}{2}x + \frac{3}{8}x^2 + \cdots
\end{align}

Differentials

\begin{gather}
    \Delta f = f(x + \Delta x) - f(x) \approx f'(x)\Delta x\\
    \deriv{f}{x} \approx \frac{\Delta f}{\Delta x}
\end{gather}

\section{Newton's Laws}

\begin{enumerate}
    \item Inertial systems exist
    \item \(\vec{F} = m \vec{a}\)
    \item \(\vec{F_{ba}} = -\vec{F_{ab}}\)
\end{enumerate}

\subsection{Ficticious Forces}

When \(\vec{R}\) is the vector from the origin of an inertial system to a new system,

\begin{gather}
    \vec{r'} = \vec{r} - R\\
    \vec{F}_\text{apparent} = \vec{F}_\text{true} - M\ddot{\vec{R}}\\
    \vec{F}_\text{apparent} = \vec{F}_\text{true} + \vec{F}_\text{fictitious}
\end{gather}

\subsection{Problem Solving Steps}

\begin{enumerate}
    \item Draw force diagrams for each mass
    \item Set up coordinates
    \item Write equations of motion (\(\sum F = Ma\))
    \item Write down constraints
\end{enumerate}

\subsubsection{Constraints}

\begin{enumerate}
    \item Rope length does not change
    \item Mass of rope is 0
    \item Normal force forms third-law pair
    \item Direction of motion
\end{enumerate}

\section{Forces}

\subsection{Gravitational Force}

\begin{gather}
    \vec{F_{ba}} = -\frac{GM_aM_b}{r^2} \uvec{r}_{ba} = +\frac{GM_aM_b}{r^2} \uvec{r}_{ab} = - \vec{F_{ab}}
\end{gather}

\subsubsection{Shell Theorem}

Gravitational force of a uniform think spherical shell of mass M and radius R experiences

\begin{enumerate}
    \item A force equivalent to that if all mass were concentrated in the center, if \(r>R\)
    \item No force if \(r<R\)
\end{enumerate}

\subsubsection{Acceleration}

\begin{gather}
    \vec{g} = -\frac{GM_e}{R_e^{\,2}}\uvec{r} \approx \qty{9.8}{\metre\per\second^2}
\end{gather}

\subsubsection{Weight}

\begin{align}
    \vec{W} =& -G\frac{M_em}{R_e^{\,2}}\uvec{r}\\
    \vec{W} =& \, m\vec{g}
\end{align}

\subsection{Electrostatic Force}

\begin{gather}
    \vec{F}_{ba} = k\frac{q_aq_b}{r^2}\uvec{r}_{ba}
\end{gather}

\subsection{Frictional Force}

For bodies not in relative motion (static):

\[0 \leq f \leq \mu N\]

For bodies in relative motion (kinetic):

\[f = \mu N\]

\subsection{Viscosity}

\begin{gather}
    \vec{F}_v = -C\vec{v}\\
    m\deriv{v}{t} = -Cv\\
    \deriv{v}{t} = -\frac{1}{\tau}v\\
    v = v_0e^{-\frac{t}{\tau}}
\end{gather}

\(\tau=\frac{m}{C}\) is a characteristic time of the system, such that after a time \(\tau\), the velocity will drop by a factor of \(\frac{1}{e} \approx 0.37\)

The body only travels a distance \(v_0\tau\)

\section{Equilibrium}

\begin{gather}
    F_n = 0\\
    \tau = 0
\end{gather}

\section{Simple Harmonic Motion}

Equation of motion is the following 2nd order differential equation

\begin{gather}
    M\deriv{^2x}{t^2} = -kx\\
    \deriv{^2x}{t^2} + \frac{k}{M}x = 0 \\
    \omega = \sqrt{\frac{k}{m}}
\end{gather}

\subsection{Hooke's Law}

\[F_s = -k(x-x_0)\]

\section{Momentum}

Newton's 2nd law using momentum

\begin{align}
    \vec{F} &= M\vec{a}\\
    \vec{F} &= \deriv{}{t}(M\vec{v})\\
    \vec{F} &= \deriv{\vec{P}}{t}
\end{align}

Dynamics of a system of particles

\begin{gather}
    \vec{F}_j = \deriv{\vec{p}_j}{t}\\
    \vec{F}_j^{int} + \vec{F}_j^{ext} = \deriv{\vec{p}_j}{t}\\
    \sum_{j=1}^{N}\vec{F}_j^{int} + \sum_{j=1}^{N}\vec{F}_j^{ext} = \sum_{j=1}^{N}\deriv{\vec{p}_j}{t}\\
    \vec{F}_{ext} = \sum_{j=1}^{N} \deriv{\vec{p}_j}{t} = \deriv{\vec{P}}{t}
\end{gather}

\subsection{Center of Mass}

\begin{gather}
    \vec{F} = \deriv{\vec{P}}{t}\\
    \vec{F} = M\ddot{\vec{R}} = \sum_{j=1}^{N}m_j\ddot{\vec{r}}_j\\
    \vec{R} = \frac{1}{M}\sum_{j=1}^{N}m_j\vec{r}_j
\end{gather}

Center of mass of an extended body

\begin{gather}
    \vec{R} = \frac{1}{M}\int_V\vec{r}\,dm\\
    \vec{R} = \frac{1}{M}\int_V\vec{r}\rho \, dV
\end{gather}

\subsection{Conservation of Momentum}

For an isolated system,

\[\vec{F} = \deriv{\vec{P}}{t} = 0\]

\subsection{Impulse}

\begin{gather}
    \int_{0}^{t}\vec{F}\, dt = \vec{P}(t) - \vec{P}(0)\\
    \vec{I} = \Delta \vec{P}
\end{gather}

\subsection{Rockets}

For a rocket of mass \(M\) moving at velocity \(\vec{v}\) expelling a mass of \(\Delta m\) at a relative velocity \(\vec{u}\)

\begin{gather}
    \vec{P}(t) = (M+\Delta m)\vec{v}\\
    \vec{P}(t+\Delta t) = M(\vec{v}+\Delta \vec{v}) + \Delta m (\vec{v} + \Delta \vec{v} + \vec{u})\\
    \Delta \vec{P} = M\Delta \vec{v} + \Delta m(\Delta \vec{v} + \vec{u})\\
    \deriv{\vec{P}}{t} = M \deriv{\vec{v}}{t} + \vec{u} \deriv{m}{t}\\
    0 = M \deriv{\vec{v}}{t} - \vec{u} \deriv{M}{t}\\
    \Delta \vec{v} = -\vec{u} \ln(\frac{M_0}{M_f})
\end{gather}

\subsection{Momentum Flow}

For one droplet moving with velocity \(v\) and stopping at your hand,

\begin{align}
    I =& \int F\,dt\\
    I =& \Delta p\\
    I =& m(v_f - v)\\
    I =& -mv
\end{align}

Therefore

\[I_{hand} = mv\]

If the droplets are separated by distance \(l\) and \(T\) is the time between collisions,

\begin{gather}
    F_{av}T = I = mv\\
    F_{av} = \frac{mv}{T} = \frac{mv^2}{l}
\end{gather}

Extending to a constant flow rate where \([\rho] = \unit{\kilogram \per \metre^3}\)

\begin{gather}
    \dot{\vec{P}} = \rho v^2 A \uvec{v}
\end{gather}

The flux density \(\vec{J}\) is defined as

\[\vec{J} = \rho v^2 \uvec{v}\]

Therefore,

\[\dot{\vec{P}} = (\vec{J} \cdot \vec{A})\uvec{v}\]

\section{Energy}

Deriving the Work-Energy Theorem

\begin{align}
    F(x) &= m\deriv{v}{t} =\\
    \int_{x_a}^{x_b} F(x)\,dx &= m\int_{x_a}^{x_b}\deriv{v}{t}dx\\
    &= m \int_{t_a}^{t_b}\deriv{v}{t}v\,dt\\
    &= m \int_{t_a}^{t_b}\deriv{}{t}(\frac{1}{2}v^2)dt\\
    &= \frac{1}{2}mv^2 \Big|_{t_a}^{t_b}\\
    W_{ba} &= K_b - K_a = \int_{r_a}^{r_b} \vec{F}\cdot dr = \int_{t_a}^{t_b} \vec{F}\cdot \vec{v} \, dt
\end{align}

\subsection{Power}

\begin{gather}
    \frac{\Delta W}{\Delta t} = \vec{F} \cdot \frac{\Delta \vec{r}}{\Delta t}\\
    \deriv{W}{t} = \vec{F} \cdot \vec{v} 
\end{gather}

\subsection{Conservative Forces}

Total mechanical energy does not change, given that only conservative forces act on the system

\begin{gather}
    \int_{r_a}^{r_b}\vec{F}\cdot d\vec{r} = -U(\vec{r}_b)+ U(\vec{r}_a) = K_b - K_a\\
    K_a + U_a = K_b + U_b
\end{gather}

Additionally,

\[\oint \vec{F} \cdot d\vec{r} = 0\]

\subsection{Potential Energy and Force}

\begin{gather}
    F(x) = -\deriv{U}{x}
\end{gather}

\subsection{Non-Conservative Forces}

\begin{gather}
    \vec{F} = \vec{F}^c + \vec{F}^{nc}\\
    W_{ba}^{tot} = -U_b + U_a + W_{ba}^{nc}\\
    -U_b + U_a + W_{ba}^{nc} = K_b - K_a\\
    (K_b + U_b) - (K_a + U_a) = W_{ba}^{nc}\\
    \Delta E = W_{ba}^{nc}
\end{gather}




\end{document}