\documentclass{notes}
\graphicspath{{../Images/}}

\fancyhead[l]{Jonathan Cheng}
\fancyhead[c]{Physics 5A}
\fancyhead[r]{September 10, 2024}
\fancyfoot[c]{Page \thepage\ of \pageref{LastPage}}

\begin{document}

\section{Newton's Laws of Motion}
Often a conflict between what is definition vs. observation

\subsection{Newton's First Law (N1)}
\label{N1}
Isolated objects move uniformly in inertial reference frames (\(\vec{a} = 0 \Leftrightarrow \vec{v} = \text{constant}\))

\begin{gather}
    \vec{a}_n = 0 \quad \Leftrightarrow \quad \vec{v} = \text{constant}
\end{gather}

\subsubsection*{Definition}
Inertial reference frame: The property that isolated objects move uniformly

\subsubsection*{Observation}
\begin{enumerate}
    \item Inertial reference frames can always be found
    \item Generic reference frames are often inertial
\end{enumerate}

\subsection{Newton's Second Law (N2)}
\label{N2}
For object A in inertial reference frames,
\begin{gather}
    \vec{F}_A = m_A \vec{a}_A
\end{gather}

\subsubsection*{Definition}
Mass: The object which can be accelerated by a force

\subsubsection*{Observation}
\begin{enumerate}
    \item \(\vec{a} \parallel \vec{F}\)
    \item Forces add as vectors
\end{enumerate}

\subsection{Newton's Third Law (N3)}


\subsubsection*{Definition}


\subsubsection*{Observation}


\end{document}