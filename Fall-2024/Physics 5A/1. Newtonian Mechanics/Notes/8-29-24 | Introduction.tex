\documentclass{notes}
\graphicspath{{../Images/}}

\fancyhead[l]{Jonathan Cheng}
\fancyhead[c]{Physics 5A}
\fancyhead[r]{August 29, 2024}
\fancyfoot[c]{Page \thepage\ of \pageref{LastPage}}

\begin{document}

\section{What is Classical Mechanics?}

Classical Mechanics (CM) - A theory (framework) for predicting the future location (trajectory) of particles given their current positions and velocities

\[x_0, \,v_0 \rightarrow x(t), \,v(t) \quad \forall \, t>0\]
\centerline{\(\implies\) deterministic ``clockwork'' universe}
\vspace*{10px}

Breaks down at:
\begin{enumerate}
    \item High Speeds (Special Relativity)
    \item Large Mass (General Relativity)
    \item Small Scale (Quantum Mechanics)
\end{enumerate}

\section{Notation}
\subsection{Degrees of Freedom (dof)}
Configuration - Set of numbers \(\{q_1, \dots, q_N\}\) required to specify the state of the mechanical system
Each \(q_i\) is a degree of freedom (dof)
\begin{enumerate}
    \item 1 dof - \(x(t)\)
    \item 2 dof - \(\vec{r}=\cv{x}{y}\)
    \item 3 dof - \(\vec{r}= \begin{bmatrix}
              x \\ y \\ z
          \end{bmatrix}\)
    \item 6 dof (Spin) - \{\(q_i\}=\{\vec{r}, \alpha, \beta, \delta\}\) Euler Angles
    \item (Deformation) - Need field \(R(\theta, \phi)\)
    \item Positions of every \(~10^{26}th\) atom
\end{enumerate}

For 1-6, Newton predicts trajectory \(q_i(t)\) from \(q_i(t=0), v_i(t=0)=\frac{dq_i}{dt}=\dot{q}_i\)

\hspace*{0.5in} \(\implies\) Physics is all about making ``things as simple as possible, but no simpler'' (Einstein)
\newpage
\section{Dimensional Analysis}

 [\(\dots\)] = Dimension of \(\dots\)

If answer is \(A=B\), check that \([A]=[B]\)

\begin{table}[h]
    \caption{Base Quantities}
    \label{Dimensional Analysis}
    \begin{center}
        \begin{tabular}{|c|c|c|c|}
            \hline
            Quantity & Length    & Time       & Mass          \\
            \hline
            SI Unit  & meter 'm' & second 's' & kilogram 'kg' \\
            \hline
            Measured & L         & T          & M             \\
            \hline
        \end{tabular}
    \end{center}
\end{table}

\end{document}

