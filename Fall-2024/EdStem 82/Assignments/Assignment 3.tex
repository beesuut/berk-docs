\documentclass{notes}
\graphicspath{{../Images/}}

\fancyhead[l]{Jonathan Cheng}
\fancyhead[c]{EdStem 82: Assignment 3}
\fancyhead[r]{December 11, 2024}
\fancyfoot[c]{Page \thepage\ of \pageref{LastPage}}

\begin{document}

\section*{Assignment 3: Unit Plan}

\subsection*{Grade Level \& Subject}

7th grade Math

\subsection*{Big Ideas \& Driving Question}

Apply and extend previous understandings of operations with fractions to add, subtract, multiply, and divide rational numbers

Understand that multiplication is extended from fractions to rational numbers by requiring that operations continue to satisfy the properties of operations, particularly the distributive property, leading to products such as (–1)(–1) = 1 and the rules for multiplying signed numbers. Interpret products of rational numbers by describing real-world contexts. (7.NS.A.2)

\textbf{\ul{What does it mean for a number to be negative?}}

\subsection*{Learning Outcomes}

\begin{enumerate}
    \item Students will identify patterns in basic mathematical operations through the use of physical (distance) interpretations
    \item Students will identify that similar patterns are found through the use of monetary interpretations of multiplications
    \item Students will engage with the interpretations and build mathematical objects using what they learned from their observations
    \item Students will discover and present their own examples of multiplying signed and rational numbers
\end{enumerate}

\subsection*{Prior Knowledge / Learning Gaps}

Students must have a concept of distance and monetary value

It may be beneficial that students understand the distributive property as well as the concept of a number line

\subsection*{Materials Needed}

\begin{enumerate}
    \item Poster Paper
    \item Illustration materials (Pens, Markers, Colored Pencils)
    \item Tape
\end{enumerate}

\subsection*{Classroom Arrangements / Safety Concerns}

Lesson 1 requires the classroom to have a `front' which is clear for walking and visible to all students

Lesson 2 and 3 require desks to be set up such that students can work \& discuss in small groups

\section{Lesson 1}

\subsection{Engage:}

Activities that engage students’ interest and build connections to their lives and prior knowledge

\ul{What will you say and do to introduce your topic and to engage students’ attention?}

To begin the lesson, I wil introduce the driving question to the students.

As a first activity, students will experience real world scenarios which imply these mathematical properties without explicitly reference to symbolic math.

I will have the students work with a number of examples such as tracking repeated changes in distance or monetary exchanges. This will give them an understanding that certain patterns remain true in the real world.

By including a number of examples, students will gain a more intuitive understanding of abstract algebraic concepts such as the distributive property and multiplication of negative numbers. In particular, it might show that multiplying negatives `cancel’ and that the distributive property is a consequence of the definition of multiplication.

\textbf{Example 1.1:}

\begin{enumerate}
    \item Place tape along front wall marked as a number line
    \item Have a student act as a number
    \item Taking a step forward adds to their `number'
    \item Taking a step backwards is subtraction
    \item If they turn around and step forward, they are adding but end up with a smaller number than they started with?
\end{enumerate}

\textbf{Example 1.2:}

\begin{enumerate}
    \item 2 students are exchanging money (focus on student 1)
    \item If student 2 does an action (give money), what operation is it? (Addition)
    \item If student 1 does an action (give money), what operation is it? (Subtraction)
    \item Now student 1 takes money. It is still subtraction, but their money increases?
\end{enumerate}

Students will discuss, in groups, patterns between the 2 examples

I will touch on the idea that a negative number is the concept of a number `backwards'

\tab Ex: Turning around; Taking vs. giving money

I will show on the number line that negatives are just counting numbers but in the opposite direction (starting from zero)

Instruct students to keep those ideas and quickly repeat the examples with a new set of students (for engagement)

Move onto new examples showing the distributive property

\textbf{Example 2.1:}

\begin{enumerate}
    \item Student takes 2 steps forward and 2 steps back (reinforce previous example)
    \item What if student took a large step forward (or backward) instead?
    \item See that the final result does not change
\end{enumerate}

\textbf{Example 2.2:}

Combine examples 1.2 and 2.1 by giving / taking 1 dollar twice vs. 2 dollars

End of Class

\subsection{Evaluate:}

Previous Experience and Baseline Learning

\subsubsection{What to Watch For:}

\ul{How to see student engagement?}

Students should be in discussion with one another throughout the process.

Discussion should be on task and likely have many questions.

\ul{What prior knowledge is expected?}

Some students will likely already know these properties whereas others may not yet have a firm grasp on multiplication or the concept of negative values.

Students should have a loose grasp on the concept of a number line.

\subsubsection{Response}

\ul{How might prior knowledge be revisited?}

If needed, I may have to revisit the concept of number lines and the definition of multiplication

Number lines will have to be taught procedurally (because of time constraints) as a list of all numbers in order. Students will have a less intuitive idea but will hopefully be supported by the lesson.

Multiplication as the idea of repeated addition can be shown as the most fundamental example of the distributive property. This may be quickly covered if not already known.

\begin{gather}
    2 \times 3 = 2 \times (1 + 1 + 1)\\
    = (2 + 2 + 2)\\
    = 6
\end{gather}

\section{Lesson 2}

\subsection{Explore:}

Hands-on / Data designed to explore ideas and to develop skills together

\ul{What specific directions and demonstrations will you use to introduce the task(s), problem(s), activities or project?}

I will ask the students to (quickly) work in groups and attempt to mathematically `formalize’ their findings on either the concept of negative numbers, or the concept of the distributive property. (Groups will swap and do peer-teaching later)

I will provide a short summary of important points to think about in order to more clearly showcase the intended `direction’ of the driving question

\subsubsection{My Input (Written on Board for Students)}

Not explaining explicitly, but giving patterns to look at, learn from, and talk about

Equation forms of each example:

\begin{align}
    1 + (-1) =& \,0  &  1 - 1 =& \,0\\
    2 + (-2) =& \,0 & 2 + 2(-1) =& \,0
\end{align}

Concept of Negative Numbers:

\begin{gather}
    2 \times (-3) = -6\\
    (-2) \times (3) = -6\\
    2 \times 3 = 6 \\
    (-2) \times (-3) = \mathord{?}
\end{gather}

Distributive \& Commutative Property:

\begin{gather}
    1 + (-1) = 0 \mathord{?  \text{  (Think about physical examples)}}\\
    (-2) \times 3 = (-2)(1 + 1 + 1)\\
    (-2) \times 3 = (-2) + (-2) + (-2)
\end{gather}

\subsubsection{Mini-Project}

Students will work to create a poster explaining the concept in their own terms (as well as providing 1 or more original examples)

\tab Allows students to visualize things in a way which makes sense to them \& use relevant examples for their peers

\tab By presenting, students will have to teach the topic, meaning they have to understand it

\tab \indicates Additionally, students will be able to ask questions of myself or one another at any time

\subsection{Evaluate:}

Focus, Involvement, Collaboration, Results, and Recording

\subsubsection{What to Watch For:}

\ul{What should students be doing to construct meaning?}

Students should be working together on their model in order to combine their perspectives.

Questions and collaborative explanations amongst students are beneficial. Join conversation only if students are off track / moving in the wrong direction.

Even in my own experience, making presentations forces me to think more critically about the topic and understand it more deeply than at a hand-wave level.

\subsubsection{Response}

\ul{What prompts might encourage observation / reasoning?}

How does each part of the equation match with the provided example? (Connecting intuitive knowledge to symbolic learning)

Does it matter the order in which you multiply? Do you expect negative numbers to change this?

Are negative numbers (symbolically) any dfferent from positive numbers?

Prompts must take into account the level of each individual student

\section{Lesson 3}

\subsection{Explain \& Elaborate:}

Students explain the phenomena they explored and discuss their different
ideas and perspectives

How will you organize students to discuss their observations and explain their thinking?

\ul{Teacher-stimulated application and clarification of concepts, skills,
attitudes, processes or terminology}

Students will spend the majority of the lesson presenting their posters to groups of the opposing topic (speed-dating style). After hearing 2 examples, the voted best of each half will present to the class. After the student presentation and time for questioning, I will fill in any gaps which may have been missed, trying to strengthen the connection between the examples and the symbols. Finally, tie concepts back to the driving question.

\tab \indicates In my experience, the connection between the intuition and symbols is the hardest part of math for many students, especially once concepts become abstract, such as with negative numbers.

\ul{What opportunity will you provide for students to apply or recognize concepts or skills in another situation?}

When making their posters, students will have the opportunity to create an example of their own, connecting their knowledge to their social and cultural experiences.

\tab By connecting these different types of learning, students will hopefully see that math is integrated into other facets of life, not separate from it. Many math examples (such as the fibonacci spiral) are interesting and in the real-world, but are often not personal to students in a meaningful way.

\ul{How will you connect to students’ previous understanding and encourage students to see the development of their ideas?}

By allowing students to create their own examples, students will see that the symbolic knowledge is intimately connected to real world examples. Some connections may seem `obvious' which is part of what makes math such a fundamental skill.

\subsection{Evaluate:}

Participation, Reporting, Debating, and Evidence-Based Reasoning

Demonstrated Understanding, Use of Skills, and Other Applications

\subsubsection{What to Watch For:}

\ul{Common misconceptions?}

Confusion between negative numbers and their connection to operators (\(+\), \(-\), \(\times\), \(\div\))

Negative numbers are `smaller' than regular numbers (size is instead measured absolutely from zero)

Patterns: 2 negatives make a positive (Only applies to multiplication, not addition)

\ul{How will you know if a student needs help with applying what they learned?}

Students will be quieter in their group because they will not have the same level of input into their presentation.

\subsubsection{Response}

\ul{How will you respond to individual reluctance to share, too much sharing, or overly critical debate?}

Students may be re-grouped for their project if they are not working well together.

Students should not stifle one another, nor allocate all the work to a single group member.

\ul{How will you support students who are having trouble applying ideas or skills?}

Students who are particularly struggling may be allowed to form slightly larger groups (adding onto a pre-existing group) or join a separate group which is specifically supported by the teacher or an aide.

\end{document}