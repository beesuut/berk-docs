\documentclass{notes}
\graphicspath{{../Images/}}
\setlength{\headheight}{14.5pt}
\addtolength{\topmargin}{-2.5pt}

\fancyhead[l]{Jonathan Cheng}
\fancyhead[c]{EdStem 82}
\fancyhead[r]{September 19, 2024}
\fancyfoot[c]{Page \thepage\ of \pageref{LastPage}}

\begin{document}
\subsubsection*{Who am I?}
\tab Hi! My name is Jonathan Cheng. I'm a Taiwanese American student originally from LA. I'm currently studying math and physics at UC Berkeley with the intention of doing research in theoretical physics. Even if that means nothing to you, I'm sure you've heard the term quantum physics, the name Einstein, or the movie Oppenheimer.

\subsubsection*{My Educational Journey}
\tab I had a fairly standard education path through the 5th grade. For middle school, I went to a further school because of its more flexible programs. I was in an accelerated math class, and to be honest, I struggled in it a lot. Even up to high school, I didn't always enjoy or understand what I was doing.

\tab In high school and through to today, I figured out better ways to learn. Rather than working on each idea or problem separately, I learned to connect the ideas that I was learning. Seeing these patterns made school feel more ``applicable to the real world.'' Rather than just solving equations, I was learning to problem solve. Rather than making up symbolism in english, I was learning to be critical of information I took in.

\tab In recent years, I've founded a nonprofit tutoring organization for students with IEP's. Particularly, that means supporting the students who could benefit most from extra help (such as those with ADHD or dyscalculia). Doing this, I've found it beneficial to talk through problem solving steps, so I try to always understand and teach the logic instead of just the answer.

\tab In school, but specifically at Berkeley, I've found asking questions to be one of the most beneficial ways of learning. It helps you make connections and understand concepts at a deeper level. Rather than simply understanding what is told to me, it allows me to solve problems of my own.

\subsubsection*{My Other Interests}
\tab Personally, I enjoy reading (mostly fiction), making music, playing racquet sports, and mountain biking. I love board games, but in terms of videogames, I primarily play Stardew Valley and Minecraft. I'm also quite deep in the fandom for the show Gravity Falls.

\subsubsection*{How To Get To Know Students?}
\tab In teaching settings, I like to make a participatory environment, which for a time would mean calling on students as a `feeler' for the room. On the first days, that might include getting names and information about them. This generally makes more sense for me, as hearing all the names and information at once is boring for students, and I am likely to forget most of it. Depending on the particular lesson plan, I may also talk to students individually while helping them work.

\end{document}